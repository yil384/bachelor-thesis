% !TeX root = ../thuthesis-example.tex

\chapter{\textbf{引言}}

\section{研究背景与意义}
操作系统作为现代计算系统中最基础的系统软件之一,其设计与实现一直是计算机系统研究中的核心课题。无论是个人计算机、服务器,还是嵌入式系统、云平台、边缘计算设备等,操作系统始终承担着资源调度、任务管理、设备驱动、文件系统、网络通信等核心职责。

从上世纪 60 年代最早期的批处理系统和时间共享系统,到今天支持虚拟化、容器化、异构计算的复杂系统架构,操作系统不断演进,以适应性能、安全性、可靠性、可扩展性等方面的挑战。\textbf{内核架构}作为操作系统设计中的关键因素,不仅决定了系统的功能边界,也深刻影响了系统性能和模块间的协作方式。

随着应用场景的多样化与硬件架构的不断演进,传统通用操作系统难以在所有环境下均达到最佳性能和安全性。为此,\textbf{定制化操作系统}应运而生,即根据特定应用需求和场景对操作系统结构与功能进行专门设计与优化,以实现性能、安全性、资源利用率等方面的提升\cite{Ammons2007Libra, Peter2015Arrakis}。定制化操作系统不仅涉及对单个组件的优化,还可能重构整体内核架构,从而催生出多种内核设计范式,如宏内核(Monolithic Kernel)、微内核(Microkernel)、库操作系统(Library OS)、Unikernel、Hypervisor 等。

宏内核架构将大部分系统服务集中在内核空间,性能优越但缺乏灵活性;微内核则精简内核功能,将非核心服务移至用户空间,增强了安全性和模块化\cite{Black1991Microkernel};库操作系统将系统服务封装为可链接库,极大提升了定制能力和资源隔离\cite{Ammons2007Libra, Madhayapeddy2013Unikernels}。Unikernel 将应用与最小化操作系统合二为一,适合云计算和边缘计算环境\cite{Madhayapeddy2013Unikernels}。Hypervisor 通过虚拟化技术支持多租户资源隔离,广泛应用于数据中心\cite{Popek1974Virtualization}。此外,混合内核(Hybrid Kernel)和多核架构(Multikernel)也在兼顾性能与模块化方面取得了进展\cite{Abdalkarim2022Hybrid, Baumann2009Multikernel}。针对高性能计算,Exokernel 提供了对硬件资源的极致控制能力\cite{Engler1995Exokernel}。

典型的定制化操作系统如 Libra、Arrakis、IX、OSv 等在各自场景均展示了优于通用操作系统的表现\cite{Ammons2007Libra, Peter2015Arrakis}。例如,Libra 基于库操作系统思想,专为虚拟化环境设计;Arrakis 依托宏内核,着重资源管理与安全;OSv 则结合了 Hypervisor 和 Exokernel 的优势,优化云服务性能。

传统的\textbf{宏内核(Monolithic Kernel)}架构将大部分功能模块集中在一个统一的内核空间中运行,这种设计便于模块间高效通信和共享数据,因此在性能上表现优越。Linux、Windows 等主流操作系统都采用了宏内核设计。然而,宏内核的缺点也逐渐暴露:内核臃肿、维护困难、模块间耦合严重,导致系统在面对快速变化的软硬件需求时缺乏灵活性。

相对地,\textbf{微内核(Microkernel)}架构则试图将操作系统最小化,仅保留进程调度、内存管理、IPC 等核心功能,其它功能如文件系统、网络协议栈等运行于内核态。这种设计显著增强了系统的安全性和模块化程度,但也由于频繁的用户态/内核态切换带来了性能瓶颈。

在性能和灵活性之间,\textbf{组件化操作系统(Component-based OS)}设计应运而生,尝试在宏内核与微内核之间寻求一种新的平衡。这类操作系统将系统核心功能划分为松耦合、高内聚的模块,各模块通过明确定义的接口协作运行,并能根据实际需求灵活组合和部署。

\textbf{ArceOS} 正是典型的组件化操作系统代表之一。它采用\textquotedblleft 库操作系统(Library OS)\textquotedblright 与宏内核思想融合的方式,将传统内核功能打散为多个组件,使得系统具有类似微内核的模块化结构,又保留了宏内核的高性能通信特性。这一架构对于构建下一代适应\textquotedblleft 多平台、多场景、多功能\textquotedblright 的操作系统具有重要意义。

尤其是在当下容器化与云原生技术快速发展的背景下,用户对操作系统提出了更高的要求——既要运行在性能受限的边缘设备上,也要能支持高度并发的分布式服务架构,还要具备运行标准 Linux 应用程序的能力。如何\textbf{在一个灵活、可裁剪的操作系统架构下,提供与 Linux 应用兼容的运行环境},成为本研究的切入点和目标。

本论文围绕\textbf{在 ArceOS 中设计与实现支持 Linux 应用的宏内核网络管理模块接口},不仅具备较强的现实需求,也具有学术研究的创新意义:
\begin{itemize}
    \item 从理论上,探索组件化架构中宏内核模型如何更好地融合标准系统调用、进程模型与网络协议栈;
    \item 从实践上,推动面向 Linux 应用的轻量级兼容层构建,增强新兴操作系统的生态兼容能力;
    \item 从工程上,提供一个高性能、可验证、具备良好抽象分层的系统实现框架,具备推广与拓展潜力。
\end{itemize}

\section{国内外研究现状}
操作系统架构设计在过去几十年内经历了三次主要范式的变迁:
\begin{enumerate}
    \item \textbf{单体式宏内核(Monolithic)}:如 UNIX、Linux 等,强调性能优先,系统功能高度内聚,所有模块运行在内核空间,适合传统服务器和桌面应用;
    \item \textbf{微内核(Microkernel)}:如 MINIX、QNX、L4 系列,强调安全性与模块隔离,适合安全性要求高或嵌入式系统;
    \item \textbf{模块化/组件化架构(Modular/Component-based)}:如 Fuchsia、Barrelfish、ArceOS,强调灵活配置、硬件多样性支持和跨平台兼容,适应\textquotedblleft 应用多样 + 硬件异构\textquotedblright 趋势。
\end{enumerate}

国外在操作系统模块化与兼容层设计方面已经取得了大量进展。比较有代表性的成果包括:
\begin{itemize}
    \item \textbf{Microsoft Drawbridge(2011)}:将 Windows 子系统封装为一个库操作系统,并运行在轻量级虚拟化容器中,实现高效兼容;
    \item \textbf{Google Fuchsia}:其 Zircon 微内核提供统一的内核服务,用户空间运行多个组件服务,借助 FIDL 接口进行解耦通信;
    \item \textbf{Unikernel 项目(如 MirageOS、IncludeOS)}:将应用与所需最小内核功能链接为一个镜像,用于轻量化部署和极致性能需求;
    \item \textbf{L4Linux}:运行于 L4 微内核上的兼容 Linux 子系统,是微内核兼容层的重要实践;
    \item \textbf{WINE 与 WSL}:分别以系统调用翻译和系统服务桥接的方式,在非 Windows/Linux 内核上运行其原生应用。
\end{itemize}

国内的研究也紧随其后。例如:
\begin{itemize}
    \item 清华大学的 \textbf{ArceOS} 项目,结合库操作系统思想与模块化调度框架,旨在构建新一代灵活的嵌入式与云端 OS;更新的 \textbf{Starry/Starry-next} 项目,探索高度裁剪、高可配置性的组件操作系统架构,支持用户自定义内核功能裁剪;
    \item 北京大学、华中科技大学在 Linux 兼容层与模块虚拟化上也开展了相关研究。
\end{itemize}

\subsection{相关理论研究现状}
\textbf{Software Dock} 是一种支持操作系统组件化的系统,它采用基于代理的软件模型来管理生产者与消费者之间的交互。该系统的设计允许操作系统模块独立开发、独立部署,并能够灵活地组合使用。通过代理机制,Software Dock 实现了模块间的解耦,使得系统可以根据需求灵活调整功能和性能。此外,Software Dock 还能够通过灵活的配置管理机制,使得不同的模块能够在不同的应用场景下优化其性能,支持跨平台的模块适配。

在组件化操作系统的研究中,\textbf{Software Dock} 提供了一种高效的模块化管理方式,可以帮助开发者将操作系统的功能按照需求进行解耦,并灵活组装不同的模块,以便更好地满足特定的需求。

\textbf{THINK} \cite{fassino2002think}是一种采用组件化编程模型的操作系统,其核心思想是通过定义模块间的交互接口,允许灵活组装组件,创建出适用于特定需求的操作系统。THINK 的设计重点在于通过抽象和接口的设计,使得操作系统能够灵活扩展,支持多种不同的应用场景。通过该模型,开发者能够快速地组装组件,实现操作系统功能的定制化,而不需要从零开始编写整个操作系统内核。

与传统的操作系统架构相比,\textbf{THINK} 的模块化设计使得系统开发更加灵活,并且能够根据需求调整系统性能。其对模块化组件的支持为后续的操作系统功能扩展提供了便利,尤其在需要针对特定应用需求进行定制的场景中,展示了其强大的可扩展性。

\textbf{OpenCom} 是一种基于组件的系统,它定义了一个最小的运行时内核,并通过模块化组件来实现系统定制。该工作特别关注跨不同硬件平台的兼容性,并且由硬件提供商提供模块加载器和链接器,开发者基于这些工具进行应用开发。OpenCom 的设计思想是简化操作系统的内核功能,采用最小化内核的策略,并通过模块化设计确保操作系统能够适应各种硬件平台,具备高度的兼容性和灵活性。

通过 OpenCom,开发者可以通过加载和卸载不同的模块来定制操作系统,并根据硬件平台的需求调整系统功能。该设计理念为硬件平台的多样性和应用需求的变化提供了极大的适应性。

\textbf{Unikraft} 是一个开源的库操作系统(Library OS),它支持定制单地址空间的操作系统内核,主要面向在主机操作系统(如 Linux)上运行的云虚拟机应用。Unikraft 旨在提供一种轻量级、高效的操作系统架构,适用于云计算环境中的虚拟机和容器化应用。通过 Unikraft,开发者可以根据具体应用的需求,定制操作系统的各个功能,并将系统功能作为库进行编译,最终形成一个高度优化的运行时环境。

Unikraft 的设计理念与 ArceOS 中的组件化架构有很多相似之处,尤其是在灵活配置和按需加载方面。两者都强调通过组件化设计来提高系统的可定制性和可扩展性,同时都在云计算和虚拟化应用中提供了优化的操作系统支持。

\textbf{FlexOS} 是一种关注配置可隔离机制的操作系统,支持多种软硬协同的隔离策略,强调操作系统的安全性和可靠性。其设计重点在于支持多种隔离策略,如资源隔离、进程隔离和网络隔离等,确保操作系统在高安全性要求的环境中能够提供稳定和可靠的服务。FlexOS 采用模块化设计,允许开发者根据实际需求选择和配置隔离策略,以确保操作系统在不同环境下的安全性和可靠性。

与 ArceOS 的设计相似,\textbf{FlexOS} 也注重操作系统的灵活配置和模块化能力。通过灵活的模块配置,FlexOS 使得开发者可以根据特定的安全性和可靠性需求定制操作系统,进而在特定应用场景中提供更加稳定和安全的服务。

\textbf{Pebble} 是为特定应用程序设计的专用操作系统,采用服务器-客户端设计,包含一个精简的内核。该系统根据领域特定的需求进一步精简操作系统的功能,确保操作系统能够针对特定的应用场景进行优化。Pebble 的设计特别关注如何在有限的硬件资源下实现高效的操作系统功能,以满足嵌入式系统和物联网设备的需求。

虽然 \textbf{Pebble} 的目标与 ArceOS 的目标有所不同,但其精简内核和定制化设计的理念与 ArceOS 的组件化和灵活配置有相似之处。两者都强调根据具体的应用需求进行操作系统的定制,以便提高系统的运行效率和满足特定功能需求。

\subsection{相关实践研究现状}
\textbf{DragonOS} 是一个完全自主研发的 64 位操作系统,专为云计算环境中的轻量化需求而设计。该系统基于 Rust 编程语言开发,具备更高的内存安全性与系统可靠性,同时兼容 Linux 二进制接口,支持虚拟化技术,并在调度子系统和设备模型等方面具备良好的扩展性。

DragonOS 采用模块化的系统架构,允许开发者根据不同的场景灵活定制所需功能模块。例如,在资源受限或虚拟化场景中,可以剔除冗余模块,仅保留关键的系统组件,从而大幅提高运行效率与系统安全性。当前 DragonOS 已实现约四分之一的 Linux 接口,其长期目标是实现对 Linux 的完全兼容,构建一个开源、可控、适用于生产环境的大规模系统平台。

与 ArceOS 类似,\textbf{DragonOS} 强调组件化设计和可插拔模块机制,具备良好的可维护性和适应性,尤其适用于需要在高安全性与高可靠性环境中运行的云端服务系统。

\textbf{Asterinas}(星绽)是一个注重内存安全性和模块隔离性的现代操作系统内核。它完全基于 Rust 编写,并严格控制 \texttt{unsafe} 代码的使用范围,仅在可信计算基础(TCB)中允许出现,以最大限度提升系统的安全性。Asterinas 使用框内核架构,提供类似 Linux 的应用二进制接口(ABI),实现对 Linux 应用的无缝兼容。

该系统支持模块化开发,提供名为 OSDK 的开发工具包,专为内核模块与驱动开发者设计。模块可以被按需加载,开发者可以自由选择模块开源或闭源,极大提升了内核开发的灵活性。Asterinas 鼓励按需定制系统组件以满足不同用户场景,从而实现高效而可靠的内核运行时支持。

\textbf{Asterinas} 与 ArceOS 在组件化思路上高度契合。两者都主张通过模块抽象与接口设计实现系统功能的可重构性,并在虚拟化场景中提供安全、快速且可维护的内核基础。

\textbf{ByteOS} 是一个支持 POSIX 接口的模块化操作系统内核,目标是提供跨平台(如 \texttt{riscv64}、\texttt{aarch64}、\texttt{x86\_64}、\texttt{loongarch64})的系统构建支持。它采用配置驱动的编译方式,用户可通过 \texttt{byteos.yaml} 配置文件定义根文件系统、文件系统实现(如 FAT32、ext4、ext4\_rs)及模块功能,最终通过 Makefile 在不同平台快速部署运行。

ByteOS 具备清晰的模块结构(\texttt{crates → arch → modules → kernel}),其中各子模块独立构建,适合进行裁剪与定制化配置。其模块化设计适用于教学、实验以及构建定制化嵌入式系统,并具备良好的移植性和平台适配能力。

与 ArceOS 的组件架构类似,\textbf{ByteOS} 强调系统功能的最小核心与外围模块的解耦,并通过统一配置系统实现快速搭建与部署,是一个典型的现代轻量级可配置操作系统平台示例。

\textbf{BrickOS} \cite{gu2024brickos} 是一个为适应多种异构平台而设计的积木式内核架构操作系统。该系统强调内核模块的高度解耦和结构清晰,通过明确定义各模块间的依赖关系,实现模块之间的边界隔离与接口协作,避免隐式状态共享问题。每个内核组件作为一个“积木单元”,可以根据需要灵活加载,构建出适配特定平台或应用场景的内核结构。

BrickOS 支持宏内核与微内核的灵活切换:在特权级下可将所有内核组件集中运行,构建类似宏内核的系统;也可将部分组件迁移至非特权级运行,实现类微内核的安全隔离。这种灵活的架构依赖于统一的硬件抽象层,使上层内核模块能够屏蔽底层资源差异,提供稳定一致的运行环境。

BrickOS 的设计包括四个关键方面:其一,内核机制的“积木化”抽象,确保各模块独立、可拼接;其二,根据平台特征与用户配置自定义系统架构与模块组合;其三,优化模块间通信延迟,提升整体性能;其四,统一异构平台的内存保护机制抽象,增强系统安全性。

\textbf{BrickOS} 与 ArceOS 在模块化思想上有高度契合点,特别是在模块解耦、按需加载与异构适配方面。其积木式内核理念为构建多场景、多平台统一支撑的系统平台提供了极具参考价值的设计路径。

\subsection{ArceOS与上述工作的比较}
相比于这些现有的组件化操作系统,ArceOS 在设计上独具特色,尤其在网络管理、协议栈的接入方式、系统调用接口的兼容性等方面做出了显著的创新。例如,ArceOS 在网络子系统的实现上通过 \texttt{axnet} 组件,提供了对多种协议栈的支持,并通过模块化设计实现了协议栈与硬件的解耦。而且,ArceOS 还采用了轮询机制和零拷贝优化技术,在保证高性能的同时,确保了系统在资源受限环境下的高效运行。

ArceOS 的组件化设计不仅关注系统功能的可扩展性和灵活性,还特别注重系统调用层的兼容性,确保能够支持现有的 Linux 应用。这使得 ArceOS 成为一个兼具高效性、灵活性和兼容性的操作系统,能够适应各种应用场景的需求,特别是在嵌入式系统和高性能计算领域。

\section{研究目标与内容}
\subsection{总体目标}
构建一个基于 ArceOS 宏内核架构的网络管理模块接口层,使其兼容主流 Linux 用户态应用(如 busybox、curl、netcat),并对接支持 lwIP/smoltcp 网络协议栈的用户程序运行环境,从而打通\textquotedblleft 系统调用 $\rightarrow$ 网络抽象 $\rightarrow$ 协议栈实现 $\rightarrow$ 硬件驱动\textquotedblright 的数据通路。

\subsection{具体目标}
\begin{itemize}
    \item 实现一套兼容 Linux 系统调用语义的 syscall 接口层,具体包括网络管理等核心部分;
    \item 设计并封装一个网络接口抽象层(NetAPI),支持底层协议栈(lwIP、smoltcp)的可插拔机制;
    \item 支持 Linux ELF 格式程序加载与运行,提供必要的进程调度和内存空间支持;
    \item 在 starry-next 架构下完成接口对接,实现跨平台运行环境;
    \item 通过功能验证与性能评估,评估系统调用延迟、内存使用、网络吞吐等指标。
\end{itemize}

\subsection{本文工作}
本研究的工作主要集中在 ArceOS 的网络管理模块接口层的设计与实现上,具体包括以下几个方面:
\begin{itemize}
    \item 设计与实现 ArceOS 的网络管理模块接口层,支持 lwIP/smoltcp 协议栈的用户态应用;
    \item 研究与实现 Linux 系统调用的兼容性接口,确保 Linux 应用能够在 ArceOS 上运行;
    \item 设计与实现网络抽象层(NetAPI),支持不同协议栈的灵活切换与集成;
    \item 在 QEMU 虚拟化平台上进行实验验证,评估系统性能与功能完整性。
    \item 分析与解决在实现过程中遇到的兼容性问题,如页表不兼容、协议栈初始化失败等。
    \item 总结实现过程中的经验教训,提出改进建议与未来研究方向。
\end{itemize}

\subsection{研究内容概述}
\begin{enumerate}
    \item \textbf{操作系统架构分析}:分析 ArceOS 的系统架构、模块调度机制以及组件间接口设计,理解其支持模块解耦与系统性能的关键点;
    \item \textbf{接口对接策略研究}:分析 starry-next 系统的初始化机制、页表结构、ELF 加载方式等,提出兼容层的集成方案;
    \item \textbf{模块设计与实现}:实现包括系统调用层、网络 IO 抽象层、信号与进程控制子系统,构建兼容 Linux 程序的运行支持;
    \item \textbf{实验评估与性能分析}:设计测试场景,通过 QEMU 等平台对模块功能完整性与性能数据进行采集与分析。
\end{enumerate}

\subsubsection{1.3.3 研究内容概述}

本研究以 ArceOS 操作系统为基础,结合 starry-next 项目中的模块,探索组件化内核架构在 Linux 应用兼容与网络协议栈集成方面的设计与实现方法。研究工作涵盖架构分析、接口适配、功能实现与性能评估四个层面,具体内容包括:

\begin{enumerate}
    \item \textbf{ArceOS 操作系统架构分析}:分析 ArceOS 的核心系统结构,包括模块调度机制、内核组件加载流程以及模块间接口设计。重点研究其如何通过组件化机制实现模块解耦、功能抽象与系统可扩展性,为后续模块对接提供理论基础与设计支撑(对应第2章)。
    
    \item \textbf{starry-next 系统组件与兼容性机制分析}:研究 starry-next 在启动流程、页表管理、地址空间划分以及 ELF 文件加载等方面的机制,分析其组件设计特点,为构建 ArceOS 与 starry-next 的适配层和兼容接口提出对接方案(对应第2章)。
    
    \item \textbf{网络协议栈的集成与对接策略}:深入分析 lwIP 和 smoltcp 两款主流轻量级网络协议栈的结构与接口风格,结合实际应用需求,设计统一的网络抽象接口,实现两者在 ArceOS 中的并行集成与可选切换机制,确保网络功能的灵活性与扩展性(对应第3章)。
    
    \item \textbf{系统调用与兼容接口的实现}:构建与 Linux 应用程序兼容的系统调用接口层,支持基本的网络 IO 接口封装,解决接口语义差异带来的适配问题(对应第4章)。
        
    \item \textbf{运行支持环境的构建与实验验证}:基于 QEMU 虚拟化平台搭建测试环境,通过加载用户态测试程序,验证系统各功能模块的正确性与兼容性,并对协议栈切换、系统调用延迟等关键性能指标进行测量与分析(对应第5章)。
    
    \item \textbf{实现过程中遇到的问题与解决方案}:在系统集成与开发过程中,梳理实际遇到的问题,例如页表不兼容、协议栈初始化失败、调用上下文切换出错等,并分析其成因,提出有效的技术解决路径与系统优化方案(对应第6章)。
\end{enumerate}

