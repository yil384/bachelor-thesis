% !TeX root = ../thuthesis.tex

\begin{translation}
    \label{cha:translation}
    
    \title{An Empirical Study of Rust-for-Linux:
    The Success, Dissatisfaction, and Compromise}
    \maketitle
    
    \tableofcontents
    \section{摘要}
    
Linux 发展已逾三十年,早已成为现代数字世界的计算基础:无论是庞大复杂的主机(如超级计算机),还是廉价轻量的嵌入式设备(如物联网设备),无数应用都构建于其之上。然而,这一基础设施自诞生之日起便一直受到内存与并发错误的困扰,其根源在于 C 语言对内存操作的宽松约束,允许大量“野蛮”的访问行为。

近期项目 Rust-for-Linux(RFL)有望一劳永逸地缓解 Linux 内核中的安全问题。RFL 通过将 Rust 的静态所有权机制与类型检查引入内核代码,使得内核在不牺牲性能的前提下,有可能摆脱内存与并发错误的困扰。尽管 RFL 正在逐步成熟并被并入 Linux 主线内核,其安全性与性能是否能够真正兼得仍未得到系统性研究。

为此,本文开展了对 RFL 的首次实证研究,旨在揭示其发展现状与带来的益处,特别关注 Rust 如何与 Linux 内核融合,以及该融合是否能在不增加系统开销的前提下保障驱动程序的安全性。我们收集并分析了六个关键的 RFL 驱动,涵盖数百个 issue 与 PR、数千次 GitHub 提交与邮件往来,以及超 1.2 万条 Zulip 论坛讨论。

研究发现:虽然 Rust 在一定程度上缓解了内核中的安全漏洞,但其能力仍不足以彻底根除此类问题。更重要的是,如果使用不当,Rust 所提供的安全保障反而可能带来显著的运行时开销与开发负担。

    \section{引言}

作为当今计算基础设施的事实标准,Linux 从未停止对长期困扰系统软件的内存和并发错误的消除 \cite{55_kernel_self_protection_2024, 61_documentation_how_linux_2024, 80_sishuai_gong_snowboard_2021, 109_duo_zhang_study_2021}。尽管社区投入大量安全加固与工程努力,漏洞仍不断涌现 \cite{15_linux_cve_caused_2022, 23_linux_cve_caused_2023, 24_linux_cve_caused_2023, 25_linux_cve_caused_2023}。究其根源,主要在于 C 语言允许不受限制地访问内存对象,Linux 内核也依赖野指针转换、裸指针操作等机制构建复杂抽象层与通用框架(如设备驱动 \cite{60_book_linux_device_2024}),以追求模块化和性能。

如何在不牺牲性能的前提下保障内存安全?Rust 语言被视为一个有前途的解决方案,有望根本缓解上述问题 \cite{13_rfc_maillist_rust_2021}。作为一种新兴的、静态且强类型的系统编程语言,Rust 承诺在无运行时开销的前提下实现安全与高性能 \cite{51_rust_programming_language_2023}。其背后依托的是所有权机制 \cite{48_ownership_mechanism_rust_2023},用于根除内存与并发错误,其核心规则包括:1)每个内存位置在任意时刻只能被一个变量独占;2)所有权可通过移动或引用的方式在线程内或线程间(通过 \texttt{Send} 和 \texttt{Sync} traits)传递;3)所有者离开作用域时,对应内存自动释放。Rust 依赖静态检查在编译期强制上述规则,从而无需使用传统的重量级内存检查器 \cite{84_thuan_huynh_memory_2006, 90_michalis_kokologiannakis_genmc_2021} 或垃圾回收器 \cite{75_stephen_blackburn_oil_2004, 85_mohamed_ismail_quantitative_2018},避免了运行时中断和不确定性延迟。

\paragraph{Rust for Linux 项目:}

正是由于 Rust 上述的安全属性,2013 年一项名为 Rust-for-Linux(RFL)的尝试性项目应运而生 \cite{3_minimal_linux_kernel_2013},如图 \ref{fig:rfl_milestones} 所示。它首次构建了一个基于内核头文件的 Rust 目标文件,并在可加载模块中调用了一个 Rust 函数输出“Hello from Rust!++”,标志着 Rust 第一次在 Linux 内核空间中发声。最初,Rust 只是 Linux 的一个“边缘工具”,主要用于编写安全关键、独立且简短的内核模块函数。2019 年,写整个内核模块的 Rust 提案开始出现 \cite{8_writing_linux_kernel_2019, 11_linux_kernel_modules_2021},该提案大胆地在上游内核中引入了对内核接口和数据结构的 Rust 封装层。2020 年的 Linux Plumbers Conference(LPC)上,该想法首次提出被广泛支持 \cite{10_barriers_in_tree_2020},并促成了 2021 年首个 RFL RFC 发布 \cite{13_rfc_maillist_rust_2021}。此后,RFL 项目迅速发展,通过 Git 仓库管理代码,引入持续集成(CI)机制确保补丁质量,交流平台也从传统邮件列表转向更活跃的 Zulip 社区。2022 年,RFL 作为实验性功能正式并入 Linux v6.1 主线版本 \cite{20_linux_merges_rust_2022}。

\begin{figure}[htbp]
    \centering
    \includegraphics[width=0.5\textwidth]{figures/rfl_milestones.jpg}
    \caption{RFL 发展历程中的重要里程碑。}
    \label{fig:rfl_milestones}
\end{figure}

RFL 推动了各种 Rust 驱动的开发,包括网络 \cite{62_e1000_nic_driver_2024}、块设备 \cite{68_null_block_driver_2024, 69_nvme_device_driver_2024}、文件系统 \cite{32_puzzlefs_buiild_vfs_2023, 36_tarfs_buiild_vfs_2023}、Android \cite{59_android_binder_driver_2024} 与 GPU 驱动 \cite{67_mac_gpu_driver_2024}。其中,一个网络驱动 \cite{40_first_rust_driver_2023} 经过 11 轮联合审查后成功并入主线 Linux v6.8,意味着 RFL 已从实验阶段逐步进入真实使用场景。尽管尚处于早期阶段,RFL 已成为最活跃的内核子系统之一 \cite{47_official_statistics_rfl_2023},可与 eBPF \cite{63_ebpf_project_2024} 和 io\_uring \cite{65_io_uring_project_2024} 并驾齐驱。

\paragraph{研究动机:}

尽管 Rust 语言特性已经被广泛研究 \cite{71_vytautas_astrauskas_how_2020, 103_boqin_qin_understanding_2020, 110_yuchen_zhang_towards_2023},Rust 与 Linux 的结合(即 RFL)仍鲜有系统性分析。理解其融合过程对于指导安全高效驱动开发具有重要意义,也为将 Rust 应用于更多内核子系统提供参考。

此外,RFL 提供了观察新语言如何融入大型传统代码库的独特视角。这一过程不仅包括将内核库封装为 Rust crate,也涉及语法语义的适配(如将 C 风格的宏泛型范式翻译为 Rust 风格),还包括将 Rust 安全规则融入对执行内存更宽松的内核环境中,以及引入新协作模式和工具链,逐步影响 Linux 的开发流程。该过程融合了成功、争议与妥协。在 2022 年以来已有超 2 万行 RFL 代码合入主线背景下,我们认为此时对其发展历程进行回顾具有现实意义。

因此,本文旨在开展对 RFL 融合过程的首次系统研究。我们收集了 RFL 项目的 6 个驱动、269 个 Issue、763 个 PR、1540 次提交、3611 条邮件与 12501 条论坛讨论内容。

\paragraph{研究问题:}

本文将围绕以下三个关键研究问题(RQ)展开:

\begin{itemize}
  \item \textbf{RQ1:RFL 的现状如何?}我们分析现有数据并深入驱动实现,发现尽管 RFL 的构建系统(如 Kbuild)已趋成熟,但缺乏主流驱动和文件系统,核心瓶颈在于代码审查进展缓慢。此外,尽管 Rust Traits 减轻了手动审计负担,但并非保障内存安全的“银弹”,Rust 与内核在内存操作的语义差异带来复杂的绕行方案,造成运行与开发开销。
  \item \textbf{RQ2:RFL 是否达成其初衷?}我们重评社区最初设定的三个目标:更安全、性能损失小、开发更容易。实测五个 RFL 驱动发现,Rust 提高了安全“可达性”但非绝对安全,unsafe 是驱动开发不可避免的选择。同时,Traits 与智能指针引发大量 icache miss,性能在部分场景不及 C;但 Rust 显著提升了代码质量,吸引了非 C 背景开发者。
  \item \textbf{RQ3:从 RFL 中可获得哪些经验?}我们提出了开发者应谨慎对待 Rust 安全保障的建议,并为社区提供一个评估“Rust 化”子系统价值的模型:通过对安全收益与开发代价的权衡评估,优先推荐将 ext4 和 linux-block 等 7 个子系统及其 25 个驱动作为推进重点。
\end{itemize}

\paragraph{贡献:}

相较于现有研究关注 Rust 安全与性能 \cite{81_amélie_gonzalez_takeaways_2023, 97_zhuohua_li_securing_2019},本文首次全面剖析 Rust 与 Linux 的融合过程,贡献如下:

\begin{itemize}
  \item 构建包含 1540 次提交、269 个 Issue、763 个 PR 与 3611 封邮件的数据集,并将在公开发表后开源。
  \item 开发自动化工具集,实现 API 统计分析与子系统视角的结果展示。
  \item 提出对 RFL 当前发展状态的深入剖析,总结将 Rust 融入内核的主要挑战,并提供针对未来驱动开发的实用建议。
\end{itemize}

完整数据与工具已发布于:https://github.com/Richardhongyu/rfl\_empirical\_tools。我们将持续追踪 RFL 项目演进并定期更新仓库内容。

    \section{Rust 在 Linux 中的基础原理}

\subsection{Rust 的安全模型}

Rust 是一种静态、强类型的编程语言,其安全模型通过对内存访问进行规范,从而保障内存和线程安全。简而言之,在任一时刻,仅允许一个变量写入某个内存位置,但多个变量可以同时读取。为此,Rust 实施了以下严格规则:

\begin{itemize}
  \item \textbf{所有权与生命周期}。Rust 要求每个内存位置(即值)在任意时间只能有一个所有者,其它变量仅可通过引用读取该值。这种机制类似于 affine 类型系统 \cite{105_aaron_weiss_oxide_2019}。所有者的生命周期定义其作用域:一旦所有者离开作用域,与其绑定的值以及其引用变量将被 Rust 自动释放。
  
  \item \textbf{移动与借用}。变量的所有权可以通过赋值或作为函数参数传递而发生转移(\textit{Move})。也可以通过生成引用在不转移所有权的前提下借用变量(\textit{Borrow})。一旦发生 Move 或 Borrow,原所有者将失去修改权限。Rust 还允许通过 \texttt{Send} 和 \texttt{Sync} traits 在多线程之间 Move 或 Borrow 所有权。
\end{itemize}

\subsection{unsafe 关键字}

尽管上述机制保证了内存与线程安全,但它也极大地限制了语言的表达能力。例如,双向链表等复杂数据结构由于每个节点需被前驱与后继同时引用(即共享所有权),便违反了所有权独占性规则。因此,Rust 引入了 \texttt{unsafe} 关键字,作为一种“逃生舱”,允许程序员绕过编译器的静态检查。

\texttt{unsafe} 代码块中,程序员可以进行多种操作,包括裸指针解引用、调用外部函数接口(FFI)函数,甚至嵌入内联汇编。Rust 编译器在这些区域内将安全性责任转交给程序员。因此,只要程序员能够通过安全 API 封装 unsafe 区域,就能确保整个程序依然维持 Rust 的安全保证 \cite{86_ralf_jung_rustbelt_2017}。

\subsection{Rust 与 Linux 的集成方式}

为将 Rust 编写的设备驱动集成进 C 编写的 Linux 内核,Rust-for-Linux(RFL)首先预处理 Rust 驱动所依赖的内核 API(如 \texttt{kmalloc})。利用 \texttt{rust-bindgen} 工具,RFL 自动生成对应的 Rust FFI 接口,这些接口符合 Rust 的调用约定,并被封装为 \textit{kernel crate}(即内核库),可通过 FFI 直接调用。

由于这些 API 最终运行于内核地址空间,无法由 Rust 编译器检查,因此 RFL 会为其包装一层安全抽象层,Rust 驱动只能调用该层导出的安全 API。例如,图~\ref{fig:rust_driver_integration} 展示了一个将 Rust 字符设备驱动集成至 Linux 的完整流程。

具体而言,\texttt{rust-bindgen} 工具根据驱动依赖的头文件(如 \texttt{cdev.h})生成三个 crate 中的 FFI 接口(图~\ref{fig:rust_driver_integration}a);开发者随后手动构造安全抽象层,将 \texttt{bindings::cdev\_alloc} 等 unsafe 调用封装至安全函数 \texttt{alloc} 中(图~\ref{fig:rust_driver_integration}b);最终,驱动仅通过调用这些抽象层 API 与内核基础设施交互(图~\ref{fig:rust_driver_integration}c)。

\begin{figure}[h]
    \centering
    \includegraphics[width=0.9\linewidth]{figures/rfl_architecture.jpg}
    \caption{Rust-for-Linux 的架构示意图。
    (a) 使用 \texttt{rust-bindgen} 生成内核数据结构与接口的 FFI 绑定;
    (b) 开发者通过封装 unsafe FFI 构建安全抽象层(kernel crate);
    (c) 驱动(drivers crate)调用安全抽象层接口以实现零开销的安全性。}
    \label{fig:rust_driver_integration}
\end{figure}
  

值得注意的是,RFL 使用注释作为开发者之间的契约,明确 unsafe 区块的前置条件与变量使用方式,以推理其安全性。

\subsection{RFL 的目标}

如表~\ref{tab:rfl_goals} 所示,RFL 在最初的 RFC \cite{13_rfc_maillist_rust_2021} 与官方展示 \cite{14_rust_for_linux_2021} 中明确提出三大目标:在不牺牲性能的前提下提升内核驱动的安全性,引入现代开发体验与高效工具链,并吸引更多开发者参与 Linux 内核开发。

\begin{table}[h]
    \centering
    \caption{RFL 项目的初始目标 \cite{13_rfc_maillist_rust_2021}}
    \begin{tabular}{p{4cm}p{9cm}}
      \toprule
      \textbf{目标维度} & \textbf{具体内容} \\
      \midrule
      安全性(Safety) & 提供内存安全与线程安全的驱动开发机制。 \\
      性能(Performance) & 在抽象层不引入额外性能开销,实现零开销。 \\
      工具链(Tools) & 提供更完善的文档与更高质量的 CI 测试支持。 \\
      开发效率(Efficiency) & 提高内核开发过程中的效率与可维护性。 \\
      社区(Community) & 吸引更多开发者参与内核社区开发。 \\
      \bottomrule
    \end{tabular}
    \label{tab:rfl_goals}
\end{table}
  
    \section{RQ1:RFL 的现状如何?}

本节首先基于所收集的提交记录、问题追踪(issues)与邮件交流,对 Rust-for-Linux(RFL)当前的开发状态进行了系统分析。随后,我们通过深入探讨 RFL 的安全抽象层与驱动实现,揭示 Rust 安全模型与内核传统编程范式之间的张力。

\subsection{RFL 开发现状}

我们的关键观察是:尽管 RFL 项目仍处于早期,但其底层基础设施(如中断 irq、内存管理 mm、调度 sched)已较为成熟;而驱动与文件系统(作为相对独立的子系统)则明显不足,预计将成为下一阶段发展重点。但由于缺乏同时熟悉 Rust 与内核开发的高质量审稿人,该进展受到显著制约。

\subsubsection{方法论}

我们通过 GitHub 上的 PR 与提交记录,以及 Linux 邮件列表中的补丁记录,收集了 RFL 项目的代码,因为 RFL 社区通常在提交内核主线之前,会先在 GitHub 上进行协作开发 \cite{37_collaboration_method_rfl_2023}。

我们根据 RFL 社区的协作模型将代码分为三个阶段:

\begin{enumerate}
  \item \textbf{待审(pending)}:仍处于 GitHub PR 中,等待首次审查;
  \item \textbf{候审(staged)}:已通过 PR 阶段,进入邮件列表供内核子系统维护者正式审查;
  \item \textbf{已合并(merged)}:正式被合并至 Linux 主线或 Linux-next 分支。
\end{enumerate}

总体而言,我们共收集到 160 余条已合并提交(约 1.9 万行代码),1300 多条候审提交(11.2 万行)与 500 多条待审提交(18.6 万行)。随后,我们使用自研工具基于正则表达式解析内核数据结构与函数接口,并按子系统与上述三阶段统计其演化与分布。

\subsubsection{结果分析}

\paragraph{(1)开发进度}

我们希望评估当前 RFL 代码库与理想中的全面 Rust 化内核之间的差距。从代码行数角度看,已合并代码仅占内核总量的约 0.125\%,而其余 92.9\% 的代码仍处于待审或候审阶段。如图~\ref{fig:wrap_progress} 所示,我们进一步将已合并代码按其所属子系统分类展示。

\textbf{洞察1:驱动、网络与文件系统构成 RFL 的“长尾”区域。}我们观察到一个明显的长尾现象:大部分 RFL 代码集中于调度器、内存管理与 IRQ 相关的基础设施。而与 Linux v6.2 中 78\% 代码量相关的驱动、文件系统、网络与安全子系统,仅覆盖极少部分。这一现象合理:基础子系统为多数驱动共享,优先支持具有更高价值;而驱动与文件系统通常与特定设备强相关,需更高编程与审查成本。例如,netdev 社区审查一个网络 PHY 驱动补丁便历经了 6 个月与 11 个版本迭代 \cite{46_network_phy_driver_2023}。

\paragraph{(2)补丁分布}

为分析 RFL 内部组成的演化过程,我们将补丁按使用场景分类为三类:Rust 编译器相关、Kbuild 系统支持与安全抽象层构建,并绘制其时间变化趋势,如图~\ref{fig:patch_distribution} 所示。

\textbf{洞察2:RFL 的基础设施趋于稳定,安全抽象与驱动将成为后续重心。}证据包括:

\begin{itemize}
  \item Kbuild 相关补丁占比逐步下降,说明构建系统已基本完成;
  \item 安全抽象层补丁占比显著提升,在 18 个月内从 20\% 提高至 60\%;
  \item 在 2023 年 4 月出现一个提交激增点,对 RFL 库进行修改以支持固定初始化对象的安全实现 —— 修复前,gpio\_pl061 与 bcm2835\_rng 驱动仍使用 \texttt{unsafe} 初始化器。
\end{itemize}

\paragraph{(3)发展趋势}

图~\ref{fig:rfl_trend} 展示了 RFL 项目的提交数与邮件活跃度随时间的演化,并标注 PR 平均审查时长。

\textbf{洞察3:RFL 的瓶颈不在代码开发,而在于审查流程。}从左图斜率可见:RFL 在初期快速增长后逐渐趋于平台化;但右图显示,PR 审查速度显著下降。例如,2023 年上半年提交的 PR 平均需 280 小时才能被审阅,是三年前的 200 倍。这说明代码产出速度远快于其被“消费”的速度(即审查与合并)。

这一现象的原因包括:

\begin{itemize}
  \item 缺乏熟悉 Rust 与内核的跨界审查者 \cite{27_envidence_rfl_community_2023,29_envidence_v4l2_community_2023};
  \item RFL 与内核子系统在协作习惯上的错位(如响应周期与风格) \cite{28_envidence_rfl_community_2023};
  \item 存在发展“死锁”:子系统社区倾向于先看到 Rust 驱动才审查抽象层,而没有抽象层又难以构建驱动 \cite{39_deadlock_rfl_abstraction_2023}。好在该问题已引起注意,并已有初步解决方案提出 \cite{34_rfl_breaks_rule_2023}。
\end{itemize}

令人鼓舞的是,RFL 正逐步被主线社区所接纳。例如,NVMe、NULL block、V4L2 与 e1000 等近期驱动,皆由内核社区主导开发并已采用 RFL 实现。

\begin{figure}[h]
  \centering
  \includegraphics[width=0.9\linewidth]{figures/wrap_progress.jpg}
  \caption{RFL 已合并的 API 封装进度概览}
  \label{fig:wrap_progress}
\end{figure}

\begin{figure}[h]
  \centering
  \includegraphics[width=0.9\linewidth]{figures/patch_distribution.jpg}
  \caption{RFL 补丁类型时间分布:Rust 编译器、Kbuild、抽象层}
  \label{fig:patch_distribution}
\end{figure}

\begin{figure}[h]
  \centering
  \includegraphics[width=0.9\linewidth]{figures/rfl_trend.jpg}
  \caption{RFL 提交与审查趋势图(左:代码提交;右:平均审查时长)}
  \label{fig:rfl_trend}
\end{figure}

\subsection{利用安全抽象实现 Linux 内核的 Rust 化}

安全抽象是实现 Linux 内核 Rust 化的关键路径之一,也是 RFL 代码量中最大的组成部分之一(见 §3.1)。顾名思义,该抽象层安全地将 C 编写的内核扩展到 Rust 驱动中:通过对内核数据结构和接口的抽象封装,使其在被调用时仍能保证内存与线程安全。

\paragraph{挑战:驯服内核编程范式}

为构建安全抽象层,RFL 首先将内核中的数据结构与接口翻译成 Rust 风格;随后,将其包装为 Rust 接口并导出给驱动使用(见 §2)。尽管该流程看似直接,但 RFL 需要应对一个关键挑战:如何将内核的编程范式与 Rust 的安全规则相协调?

例如,Linux 内核广泛使用类型转换、指针运算与位操作等技巧,这些都与 Rust 的理念存在冲突。为解决这一矛盾,RFL 采用了一系列规避策略,系统化地将内核状态纳入 Rust 风格管理体系,甚至推动语言自身演进以适应内核环境。

\paragraph{内核数据结构转换}

RFL 借助 \texttt{bindgen} 工具自动生成内核 C 结构体在 Rust 中的绑定。该过程基于规则驱动与语法转换机制,将 C 类型与符号翻译为其对应的 Rust 表达式。如 C 中的 \texttt{uint32\_t} 会被映射为 \texttt{core::ffi::c\_uint},该类型是 Rust 原语类型 \texttt{u32} 的别名。该过程由 rust-bindgen 执行,遵循表~\ref{tab:bindgen_rules} 所列转换规则。

但并非所有 C 类型都能被一一映射为 Rust 原语。我们发现这种不兼容主要体现在内核中为强控内存布局而使用的语言特性上。

\textbf{洞察4:内核对内存布局精细化控制的主动设计与 Rust 哲学冲突,带来额外开销。}

\begin{itemize}
  \item \textbf{模拟位字段与联合体}。内核常用位字段和联合体提升内存效率,如 e1000 驱动用一个字存储多个标志位。但位操作违反 Rust 内存安全原则,因此无法被直接支持。RFL 使用字节数组模拟位字段,访问器方法负责读写位。虽然 Rust 支持 union,但无法提供与 C union 的 ABI 兼容性。因此,RFL 自定义一个 \texttt{\_\_BindgenUnionField} 结构体,并用 \texttt{transmute} 操作在运行时重解释内存,全部位于 \texttt{unsafe} 块中。这些模拟带来的主要开销体现在二进制体积增加(见 §4.2)。
  
  \item \textbf{属性支持不完整}。为提升局部性,内核结构常使用 \texttt{packed} 与 \texttt{aligned} 等属性。例如,\texttt{task\_struct} 将调度信息按 cache line 对齐。尽管 Rust 提供 \texttt{\#repr(C)} 支持,但仍可能错误处理某些属性,导致 \texttt{bindgen} 生成错误代码 \cite{7_bindgen_does_not_2019, 9_bindgen_mishandles_aligned_2020}。对于某些非常用属性,RFL 尚不支持,如 BTF 标记 \cite{21_rfl_discards_btf_2022}。而如 \texttt{randomize\_layout},虽被内核用于缓解内存攻击,RFL 则忽略该属性,因为 Rust 的所有权机制与边界检查已能规避该类问题 \cite{58_address_space_layout_2024}。
\end{itemize}

尽管生成的绑定与 C 结构体在数据布局上等价,但因其大量使用裸指针(\texttt{*mut}),无法直接向 Rust 驱动暴露。为此,RFL 使用辅助类型对绑定结构进行封装,并嵌入 Rust 风格的语义。

\textbf{洞察5:RFL 使用辅助类型将内核数据的管理托管给 Rust,同时操作逻辑仍由内核负责。}

具体而言,RFL 借助以下两种 Rust 特性实现内核数据结构的安全管理:

\begin{itemize}
  \item \textbf{类型与 \texttt{Deref} 强制转换}。RFL 在嵌入内核结构体时重载其内存访问路径,将裸指针操作转化为类型安全的访问。举例而言,设备结构中常含 \texttt{void *} 指针指向设备私有数据,C 中运行时类型转换频繁。为此,RFL 为这些结构体实现 \texttt{Deref} trait,使解引用自动转化为正确类型。
  
  \item \textbf{生命周期自动管理}。RFL 引入三种低层类型(\texttt{ScopeGuard}、\texttt{ARef}、\texttt{opaque}),绑定至辅助类型上,实现生命周期机制。例如,\texttt{ScopeGuard} 在作用域结束时通过 Drop 自动释放资源,\texttt{ARef} 在引用计数变动时自动执行增减逻辑。相比传统内核开发中需显式调用 \texttt{get\_task} / \texttt{put\_task},Rust 通过这些机制固化了非文档化的编程约定。
\end{itemize}

例如,在互斥锁或自旋锁中,RFL 重写其后端实现逻辑:在管理侧使用 \texttt{Deref} 转换访问锁保护数据,利用 \texttt{ScopeGuard} 进行内存回收;在操作侧仍调用原内核锁函数。

\paragraph{内核函数包装}

内核函数的绑定与结构体转换使用类似规则。随后,RFL 在安全抽象层中广泛使用 Rust traits 将函数包装为结构体成员或 trait 方法,具体包括:

\begin{itemize}
  \item \textbf{函数成员归类结构体中}。将某一类结构体相关函数归类为其成员,形成 OOP 风格。例如,\texttt{queue\_work\_on} 与 \texttt{\_\_INIT\_WORK\_WITH\_KEY} 被集成为 \texttt{Queue} 结构体的方法,增强代码可读性,并可避免空指针传参问题。
  
  \item \textbf{函数指针建模为 trait}。内核中大量回调函数以函数指针形式存在,RFL 将其定义为辅助类型的 trait,并指定绑定类型与所有者结构,避免类型不匹配引发的漏洞 \cite{16_linux_cve_caused_2022}。
  
  \item \textbf{包装内联函数与宏}。内核驱动广泛使用内联函数与宏(如 \texttt{for\_each\_online\_cpu})。RFL 使用非内联 Rust 函数包装内联 C 函数,因当前 Rust 对 C FFI 的内联支持不友好,虽然可实现但不推荐 \cite{44_lto_optimization_rust_2023}。而函数宏则更倾向于使用辅助函数而非 Rust 宏重写,主要因为 RFL 不希望维护两套易变的接口 \cite{82_liwei_guo_transkernel_2019}。
\end{itemize}

\begin{table}[h]
    \centering
    \caption{rust-bindgen 中 C 到 Rust 的翻译规则}
    \begin{tabular}{p{2.5cm}p{4.5cm}p{5cm}}
      \toprule
      \textbf{类别} & \textbf{C 中的定义} & \textbf{Rust 映射} \\
      \midrule
      类型 & \texttt{foo} & \texttt{core::ffi::c\_foo} \\
      类型 & \texttt{foo *} & \texttt{*mut foo} \\
      \multirow{3}{*}{属性} 
            & \texttt{aligned} & \texttt{\#repr(C)}(有 caveats \cite{7_bindgen_does_not_2019,9_bindgen_mishandles_aligned_2020}) \\
            & \texttt{unused}  & 忽略 \\
            & \texttt{weak}    & 忽略 \\
      属性 & \texttt{randomize\_layout} & 忽略(因已由 Rust 所有权语义处理) \\
      函数指针 & \texttt{fn} & \texttt{Option<fn>} \\
      \bottomrule
    \end{tabular}
    \label{tab:bindgen_rules}
\end{table}  

\subsection{使用 Rust 编写设备驱动}

随着编程范式从 C 语言转向 Rust,设备驱动开发中的“数据布局”也由传统的结构对齐逻辑转变为“数据所有权”视角的推理。在本节中,我们将从开发者视角出发,描述这种范式迁移对驱动开发的具体影响。

\paragraph{设备探测(Device Probing)}

开发者需实现设备探测回调函数,在其中分配内核资源、初始化设备数据,并在必要时注册中断处理函数。相较于 C 写法,Rust 驱动的显著不同在于,开发者必须为设备数据显式标注所有权类型,即明确哪些实体可以访问这些数据。

例如:若数据可能被多个线程共享(如加锁结构),开发者需要用 \texttt{Arc} 包裹;若希望数据不可移动以支持与 C 接口共享,则需使用 \texttt{Pin}。更复杂的是,这些所有权标注往往是嵌套的,难以一目了然。

例如:e1000 网卡驱动中的接收环形缓冲区包含一个 \texttt{RxDesc} 描述符数组,并受自旋锁保护,其类型为:
\[
\texttt{Pin<Box<SpinLock<Box<Ring<RxDesc>>>>>>}
\]
每一层嵌套类型都需调用对应的初始化器,如 \texttt{Pin::from}、\texttt{Box::try\_new} 等 \cite{62_e1000_nic_driver_2024}。

\paragraph{实现驱动函数}

接下来,开发者需要实现驱动框架所要求的函数,例如网卡驱动需实现 \texttt{ndo\_open} 与 \texttt{ndo\_start\_xmit} 回调函数(属于 \texttt{net\_device\_ops})。这一步在流程上类似于传统 C 开发,即通过硬件手册中描述的行为逻辑,开发者完成设备驱动的核心操作逻辑。

然而,Rust 在内核空间的安全规则也带来了一些新的实现挑战:

\begin{itemize}
  \item \textbf{(1)动态数组实现复杂}。C 驱动常用裸指针实现可变长数组,用于存储动态内核对象(如 pages)。而在 Rust 驱动中,开发者需引入多个封装层并实现 \texttt{dyn\_num} trait 以支持动态数量。如图~\ref{fig:dyn_array} 所示的 RROS 示例中,为实现等价的动态数组,Rust 实现增加了 83\% 的代码行数,显著膨胀了目标文件体积 \cite{96_hongyu_li_rros_2023}。
  
  \item \textbf{(2)内核上下文问题仍需人工判断}。Rust 的安全规则并不自动检查执行上下文(如原子上下文与可睡眠上下文),无论是在编译时还是运行时。因此,调用方的上下文类型仍需由开发者自行判断与保证线程安全。
\end{itemize}

\paragraph{设备清理}

当内核卸载设备或初始化过程出错时,驱动需释放已分配的资源。传统 C 驱动普遍采用 \texttt{goto} 跳转到统一的资源清理位置。而在 Rust 驱动中,这一过程可由 \texttt{Drop} trait 自动完成。通过 RFL 安全抽象层的封装,Rust 能在生命周期结束或错误发生时自动回收相关资源,显著减少了人工管理负担。

\textbf{洞察6:使用 Rust 编写安全驱动的最大挑战,在于调和 Rust 的静态、不可变规则与内核编程的灵活性,这种矛盾往往被 RFL 与 Linux 社区所忽视。}

\begin{figure}[h]
  \centering
  \includegraphics[width=0.9\linewidth]{figures/dyn_array_example.jpg}
  \caption{在 RFL 驱动中实现动态数组的复杂性示例 \cite{96_hongyu_li_rros_2023}}
  \label{fig:dyn_array}
\end{figure}

    \section{RQ2:Rust-for-Linux 是否达到了预期?}

在本节中,我们回顾社区在启动 RFL 时设定的初始目标(见表~\ref{tab:rfl_goals}),并围绕以下三个问题展开分析:(1) Rust 是否让 Linux 更加安全?(§~4.1)(2) Rust 是否引入额外开销?(§~4.2)(3) Rust 如何提升 Linux 开发体验?(§~4.3)

\subsection{RFL提升了Linux的可保障性(securability)}

\textbf{方法论}:我们聚焦于Rust语言在RFL及其驱动中的bug报告与unsafe代码块的使用情况。我们的基本逻辑是,RFL的安全性依赖于Rust语言的安全保障能力,而这具体取决于驱动程序中是否消除了unsafe代码块,以及是否提供了安全抽象API。因此,如果RFL存在安全漏洞,这些位置将是关键切入点。为此,我们首先收集了已合入与待合入RFL代码中的所有bug报告与与安全相关的代码审查内容(如第\ref{sec:classification}节所述),并依据RFL GitHub项目中使用的issue标签\cite{66_issue_labels_rust_2024}将bug划分为“编译类bug”与“健壮性(soundness)bug”;其中,死锁\cite{53_deadlock_bug_rfl_2024}与原子上下文中发生睡眠的并发类bug\cite{18_rfl_driver_bug_2022}被归入健壮性bug。随后,我们对上游仓库中的所有RFL驱动程序和Rust内核crate\cite{22_rust_for_linux_2022}进行了静态分析,重点统计其unsafe代码块的使用情况。

\textbf{结果}:我们共发现25个来自RFL已合入与待合入代码的bug,其中15个位于Linux主线,10个位于RFL阶段性分支,详见表~\ref{tab:bugs-in-rfl}。在已合入主线的bug中,11个为编译类bug,其余4个与安全抽象有关。前者多由内核构建配置失误、Clang工具链版本不兼容、Kbuild与rustc之间的不匹配所致\cite{17_rfl_bug_misconfig_2022},通常不引发实际安全风险。健壮性bug中,有6个位于安全抽象层,造成内存安全破坏;3个破坏线程安全。

我们未在Linux主线中发现RFL驱动包含unsafe代码,因为截至目前,仅有一个约130行的小型驱动被合入\cite{50_rust_implementation_drivers_2023}。但我们在提交至RFL邮件列表的驱动中,发现了若干unsafe使用情况,见表~\ref{tab:unsafe-usage-drivers}。

\textbf{后续分析}:通过对上述bug与unsafe代码的审计,我们总结了以下三个关于RFL安全性的核心发现:

\textbf{启示7}:RFL确实提升了Linux的可保障性(securability),但仍无法实现完全安全(security)。

\begin{itemize}
  \item (1) Rust的安全机制构成了内核安全的支柱。该语言层的支持可帮助开发者修复现有bug并规避潜在的内存与并发问题。我们将在第5节进一步展开说明。作为一门现代语言,Rust拥有丰富的类型系统,支持如\texttt{klint}\cite{31_lints_kernel_embedded_2023}与RustBelt\cite{86_ralf_jung_rustbelt_2017}等形式化工具用于强化内核安全。相比于C语言,RFL极大压缩了由内存错误导致的攻击面,开发者在进行安全推理时的负担显著减少。
  
  \item (2) unsafe是无法避免的,尽管漏洞是可选的。我们的审计显示,所有主要驱动中普遍存在unsafe代码块,其根源在于两个方面:首先,内核需要直接控制内存与硬件,必须绕过Rust的所有权检查。例如,使用内联汇编管理TLB与发出内存屏障\cite{67_mac_gpu_driver_2024},或解引用MMIO寄存器的裸指针等,这些操作超出了Rust所有权机制(即仿仿类型系统)的覆盖范围;其次,社区在部分API设计上不得不暂时妥协。例如,pin-init接口的内存初始化机制被Rust安全检查工具判定为unsafe,但在安全抽象层中长期存在,直到多轮资深开发者辩论后才被修复\cite{12_safe_initialization_pinned_2021,57_7_pin_init_2024,49_patch_pin_init_2023}。我们承认unsafe不等于漏洞,但它确实涉及可能的漏洞源(如MMIO),无法完全消除。

  \item (3) Bug不会消失,只是被“隐藏得更深”。一方面,被安全抽象与Rust驱动调用的内核函数仍可能存在被利用的bug;另一方面,虽然Rust编译器可以即时检测内存类bug,却无法发现语义类bug。此类bug往往源自Rust与内核在内存分配策略上的差异,难以检测,只能由精通Rust与内核的专家识别。例如,C语言实现的binder驱动存在use-after-free漏洞\cite{33_rewriting_drivers_rfl_2023},若用Rust重写后,该漏洞不会以相同形式重现,而是转变为地址映射错误,逃过Rust编译器的所有检查。
\end{itemize}

\begin{table}[htbp]
  \centering
  \caption{RFL驱动中的unsafe使用情况统计}
  \label{tab:unsafe-usage-drivers}
  \begin{tabular}{lccc}
    \toprule
    驱动名称 & 因驱动逻辑过于复杂 & 缺乏安全抽象支持 & 总数 \\
    \midrule
    GPU\cite{67_mac_gpu_driver_2024}         & 107 & 7  & 114 \\
    NVMe\cite{69_nvme_device_driver_2024}    & 44  & 16 & 60 \\
    Null block\cite{68_null_block_driver_2024} & 0   & 0  & 0  \\
    E1000\cite{62_e1000_nic_driver_2024}      & 4   & 2  & 6  \\
    Binder\cite{59_android_binder_driver_2024} & 45  & 9  & 54 \\
    GPIO\cite{64_gpio_driver_written_2024}    & 0   & 3  & 3  \\
    Semaphore\cite{70_semaphore_driver_written_2024} & 0 & 4 & 4  \\
    \bottomrule
  \end{tabular}
\end{table}

\begin{table}[htbp]
  \centering
  \caption{我们在RFL中发现的bug统计(括号内为合入/阶段性代码数量)}
  \label{tab:bugs-in-rfl}
  \begin{tabular}{lcc}
    \toprule
    来源 & 编译类bug & 健壮性bug \\
    \midrule
    GitHub\cite{22_rust_for_linux_2022}     & 4(1/3) & 7(3/4) \\
    Intel LKP\cite{42_intel_linux_kernel_2023} & 8(6/2) & 0      \\
    邮件列表\cite{45_mainling_list_rust_2023} & 4(4/0) & 2(1/1) \\
    \bottomrule
  \end{tabular}
\end{table}

\subsection{RFL 是否引入额外开销?}

本节比较了 RFL 驱动与 C 语言实现的原生内核驱动,评估其性能和体积开销。我们的研究发现,Rust 驱动的可执行文件体积显著增加;其运行时性能与 C 相当,但在不同驱动与配置之间表现差异较大,波动明显。

\textbf{实验设置}:我们遍历了 RFL 的所有 Pull Request 与代码仓库,选取了 4 个具有实际应用场景的驱动,涵盖多种 IO 功能(如网络、存储),其中 NVME 与 Binder 被认为是首批合入 Linux 主线的 Rust 驱动;另外选取了 2 个 RFL 阶段性分支中的“玩具驱动”(即 gpio 与 semaphore),它们常用于探索 Rust 与 C 实现间的差异。

在这 6 个被测驱动中,仅 e1000 与两个玩具驱动实现了与 C 驱动功能一致的完整特性,其余驱动仅实现了部分功能。对于每个具有 C 对应版本的 Rust 驱动,我们分别编译二者、对比其二进制体积,并依据前人工作\cite{97_zhuohua_li_securing_2019}进行基准测试,代表其典型负载场景。表~\ref{tab:benchmarks-rfl} 展示了实验配置。

\textbf{二进制体积对比}:如图~\ref{fig:driver-size-comparison} 所示,功能完备的 Rust 驱动二进制体积明显大于 C 实现:Binder 增长 1.2 倍,Gpio 增长 2.4 倍,Semaphore 增长 1.9 倍。我们进一步分析 `.text` 段后发现,Rust 为支持泛型编程、边界检查、生命周期管理等特性而引入了 99\% 的额外代码开销。即便是仅调用内核函数的简单封装,Rust 也会将其体积扩大约 33\%。

值得注意的是,Rust 实现的 Binder 驱动体积开销相对较小,原因在于其广泛使用了带 \texttt{unsafe} 的函数指针而非泛型编程,从而在一定程度上牺牲了 RFL 设计初衷中的安全性。

虽然部署前一般会剥离调试信息,但在嵌入式系统中资源有限,调试模式下的开销亦不容忽视。我们将在附录 §B 中进一步讨论。

\textbf{性能表现}:总体来看,Rust 驱动与 C 驱动性能大致相当,差距在 20\% 范围内。但部分场景中,Rust 明显逊色;而在另一些配置下,Rust 表现反而优于 C。

具体分析如下:

\begin{itemize}
  \item 对于 \texttt{e1000},Rust 驱动在 ping 延迟测试中性能是 C 驱动的 11 倍\textbf{慢},如图~\ref{fig:e1000-latency} 所示。主要原因在于 Rust 驱动未实现如预取(prefetch)等加速特性。
  \item 对于 \texttt{binder},Rust 与 C 在 ping 延迟上表现相近,仅有 10\% 性能差距。
  \item 对于存储驱动(如 \texttt{NVME} 与 \texttt{NULL block}),Rust 驱动在不同配置下出现了最高 61\% 的性能下降,也有最多 67\% 的性能提升,如图~\ref{fig:nullblock-comparison} 与图~\ref{fig:nvme-comparison} 所示。我们观察到 Rust 驱动更适应较小任务数与 block size,这可能与 Rust 结构体尺寸更小、更易适配缓存行有关。
\end{itemize}

为了进一步分析相同行为路径下 Rust 与 C 驱动性能差异的根因,我们利用 \texttt{vtune} 与 \texttt{ftrace} 等内核工具开展微基准测试。总结如下:

\textbf{为何 Rust 驱动可能表现更差?}
\begin{itemize}
  \item Rust 驱动的锁粒度较粗。虽然 Rust 通过语言规则保障线程安全,但高性能并发编程仍需开发者手动调优。
  \item Rust 的数组访问引入了边界检查,增加运行时开销,与已有研究结果一致,后者指出 Rust 程序在内存密集型场景下开销可能高达 C 的 2.49 倍\cite{110_yuchen_zhang_towards_2023}。
  \item Rust 采用模拟位域方式访问字段(见第 \ref{sec:bitfield} 节),结合运行时检查进一步拖慢执行速度。
  \item Rust 广泛使用智能指针共享所有权,导致缓存/TLB/分支预测失效率上升。
\end{itemize}

\textbf{为何 Rust 驱动可能表现更好?}
\begin{itemize}
  \item Rust 结构体通常比 C 更小,因其使用智能指针而非直接在结构体中分配字段内存。使用 \texttt{pahole} 工具可发现,Rust 结构体更节省缓存行。
  \item 某些 Rust 驱动未实现全部功能,导致部分代码路径被省略。
\end{itemize}

\textbf{启示8}:性能从来不是免费的——程序员的实现才是决定性因素!

\begin{table}[htbp]
  \centering
  \caption{测试 Rust/C 驱动的基准任务与指标(PC 配置:Intel i5-4590/4核,Q87主板,32GB DDR3,Samsung SSD 850 Evo,WD SN770,Intel 82545 网卡)}
  \label{tab:benchmarks-rfl}
  \begin{tabular}{lccc}
    \toprule
    驱动 & 基准工具 & 测量指标 & 测试设备 \\
    \midrule
    NVME         & fio  & 吞吐量、驱动体积 & PC \\
    Null Block   & fio  & 吞吐量             & PC \\
    E1000        & ping & 延迟               & PC \\
    Binder       & ping & 延迟               & Raspberry Pi 4B \\
    Gpio\_pl061  & -    & -                 & - \\
    Semaphore    & -    & -                 & - \\
    \bottomrule
  \end{tabular}
\end{table}

\begin{figure}[htbp]
  \centering
  \includegraphics[width=0.7\linewidth]{driver_size_comparison.jpg}
  \caption{Rust 与 C 驱动的体积比较。带 * 的表示 Rust 驱动未实现完整功能}
  \label{fig:driver-size-comparison}
\end{figure}

\begin{figure}[htbp]
  \centering
  \includegraphics[width=0.7\linewidth]{e1000_latency.jpg}
  \caption{Rust 与 C 驱动延迟对比,e1000 驱动因缺失优化特性(如预取)性能落后}
  \label{fig:e1000-latency}
\end{figure}

\begin{figure}[htbp]
  \centering
  \includegraphics[width=0.7\linewidth]{nullblock_perf.jpg}
  \caption{NULL block 驱动性能对比。绿色表示 Rust 表现更优,红色表示劣于 C}
  \label{fig:nullblock-comparison}
\end{figure}

\begin{figure}[htbp]
  \centering
  \includegraphics[width=0.7\linewidth]{nvme_perf.jpg}
  \caption{NVME 驱动性能对比}
  \label{fig:nvme-comparison}
\end{figure}

\subsection{Rust 如何改善 Linux 的开发?}

本节展示了 Rust 在提升内核代码质量、可读性以及吸引新开发者参与方面的显著作用。

\textbf{代码质量与可读性的提升}:我们采用文档覆盖率与每千行代码的持续集成(CI)错误数作为衡量软件质量的两个关键指标\cite{76_vikas_chomal_significance_2014,106_jorge_wong-mozqueda_code_2015}。

在文档覆盖率方面,我们统计了通过 \texttt{EXPORT\_SYMBOL} 与 \texttt{EXPORT\_SYMBOL\_GPL} 导出的 API;根据内核文档规范\cite{43_linux_kernel_docuement_2023},这些 API 应当全部配有文档。在 CI 错误方面,我们收集了来自 Intel LKP、Syzbot\cite{41_google_syzbot_kernel_2023} 与 KernelCI\cite{38_community_based_distributed_2023} 的相关报告。

我们将 RFL 与两个近年来发展迅速的内核子系统 eBPF 与 io\_uring 进行了比较。结果如表~\ref{tab:code-quality-metrics} 所示,RFL 在代码文档覆盖率方面表现优异(100\%),而 eBPF 与 io\_uring 仅分别为 15\% 与 31\%。在每万行代码的 CI 错误数方面,RFL 也远低于其他两个子系统,分别减少了 49\% 与 68\%。

造成这一改善的原因主要有两点:

首先,RFL 利用了 Rust 的 \texttt{rustdoc lints} 功能,强制要求为所有(或部分)接口撰写文档。相比之下,传统内核开发主要依赖默认约定或人工审查,导致文档常被省略\cite{4_linus_explained_docuementation_2015}。例如,\texttt{io\_uring\_cmd\_complete\_in\_task} 是 io\_uring 中通过 \texttt{EXPORT\_SYMBOL\_GPL} 导出的关键函数,用于在工作线程中异步完成 IO 操作,但由于缺乏文档,开发者常将其与 \texttt{io\_uring\_cmd\_done} 混淆,进而引发 AB-BA 死锁\cite{30_fix_ab_ba_2023}。事实上,社区早已将“文档缺失”视为 Linux 中的第一号 bug\cite{2_carla_schroder_missing_2009}。

其次,Rust 内建的测试机制允许开发者在提交 PR 之前便运行测试。相比之下,现有的 Linux 内核测试框架(如 KUnit\cite{26_unit_testing_framework_2023} 与 Intel LKP\cite{42_intel_linux_kernel_2023})通常在代码合入或临近合入阶段才被触发。而在 RFL 中,测试代码可通过 \texttt{\#cfg(test)} 注解集成,结合 GitHub CI,可在每次 PR 提交时自动运行。正是由于测试环节提前介入,使得 RFL 的 CI 错误显著减少,代码质量也随之提升。

\textbf{更多“新鲜血液”参与 Linux 社区开发}:我们进一步比较了 RFL 与 eBPF、io\_uring 以及传统子系统 netdev 的开发者构成。受到 netdev 相关研究启发\cite{52_time_since_first_2023},我们使用开发者首次 commit 的时间跨度作为经验指标,将开发者划分为新手(0--24 个月)、熟练开发者(24--120 个月)与资深开发者(120 个月以上)。

采样区间为 Linux 版本 6.1 至 6.4。结果如图~\ref{fig:developer-experience-dist} 所示,RFL 拥有最高比例的新手开发者(58\%),显著高于 eBPF(39\%)、io\_uring(38\%)与 netdev(29\%)。更有趣的是,我们发现其中有 29 位开发者此前从未向 Linux 提交过一行 C 代码。这表明,Rust 很可能是吸引他们进入内核社区的主要动因,打破了长期以来内核开发以 C 为中心的格局。

\textbf{然而,这些新手开发者尚未成为核心开发者或维护者}:尽管有更多新手被 Rust 吸引进入 Linux 社区,我们发现他们的贡献主要集中在 Rust 工具链或 crate 的构建上,并未实质参与内核功能开发。相较之下,第~\ref{tab:benchmarks-rfl} 表中列出的 6 个驱动中,有 5 个主要由传统 Linux 社区的资深开发者完成。这说明年轻开发者与资深开发者之间仍存在断层,Rust 语言虽提升了吸引力,但并未真正降低内核开发的门槛。

\begin{table}[htbp]
  \centering
  \caption{代码质量测量结果。文档覆盖率以 \% 表示,CI 错误数按每 1 万行代码统计}
  \label{tab:code-quality-metrics}
  \begin{tabular}{lcc}
    \toprule
    子系统     & 文档覆盖率(\%) & CI 错误数/10K LoC \\
    \midrule
    RFL        & 100\%   & 3.8 \\
    eBPF       & 15\%    & 7.5 \\
    io\_uring  & 31\%    & 11.9 \\
    \bottomrule
  \end{tabular}
\end{table}

\begin{figure}[htbp]
  \centering
  \includegraphics[width=0.8\linewidth]{developer_experience_distribution.jpg}
  \caption{RFL 与其他主流内核子系统的开发者经验分布。RFL 拥有最高比例的新手开发者}
  \label{fig:developer-experience-dist}
\end{figure}


    \section{RQ3:有哪些经验与教训?}

本节总结了我们在本次实证研究中获得的关键经验与教训。

\subsection*{面向 RFL 开发者的经验教训}

为提升 RFL 及其 Rust 驱动的安全性与实用性,开发者可考虑如下建议:

\begin{enumerate}
  \item \textbf{不要将 Rust 内建的安全检查器视为万能工具}。正如第 \ref{sec:security-analysis} 节所示,Rust 的安全检查器在真实驱动中无法识别某些语义级 bug。因此,建议开发者结合更强大的分析工具(如 RustBelt\cite{86_ralf_jung_rustbelt_2017} 与 miri\cite{54_interpreter_rust_mid_level_2024})以弥补 Rust 内建机制的不足。

  \item \textbf{从“所有权”视角构建安全的内核抽象,并管理驱动资源}。这与传统内核编程中以“内存”视角出发的方式不同。编程前应先明确结构体间的所有权关系,否则在使用智能指针错误管理所有权后,修复代价将极为高昂。

  \item \textbf{接受 unsafe 作为最后的选项}。Rust 的安全规则在处理内核内存操作时,常需大量使用智能指针与泛型编程,导致内存开销巨大(详见第 \ref{sec:performance-analysis} 节)。此时,如若经过充分审查,可选择使用 unsafe 实现以换取实用性。
\end{enumerate}

\subsection*{面向 RFL 社区的建议:如何扩展 RFL 的应用范围}

决定将 RFL 优先应用于哪些未来内核驱动或子系统至关重要,因为这需要大量的开发资源与长期维护承诺。为此,我们构建了一个效益模型,用于评估“Rust 化(Rustify)”的优先级。

该效益定义为:某驱动或子系统中可通过 Rust 机制修复的累计 bug/漏洞数量与其代码规模(LoC)的比值。直观理解是:一个规模较小但含有大量内存/线程类 bug 的子系统,其 Rust 化的价值更高。这一思路已得到社区共识的支持\cite{35_rfl_driver_selection_2023}。

为量化上述指标,我们遍历了各驱动的 git 历史,收集其至今出现的所有安全问题,并手动审查其中已修复的 bug,筛选出那些可通过 Rust 安全机制避免的内存/线程类 bug(依据第 \ref{sec:rust-safety-mechanism} 节定义)。我们共分析了来自 79 个不同子系统的 2500 多个驱动,并将分析结果绘制于图~\ref{fig:subsystem-priority}。

平均而言,每个子系统包含 1.3 个 bug/每千行代码;但在不同子系统间,该比值差异显著。其中,\texttt{linux-block} 子系统由于每行代码含最多 bug(共 438 个修复中包括 113 个数据竞争 bug 与 98 个悬垂指针 bug),因此被我们建议优先 Rust 化。令人欣喜的是,该子系统的 \texttt{null block} 驱动已经被社区用 RFL 重写,并在第 \ref{sec:performance-analysis} 节进行了测试验证。

除 \texttt{linux-block} 外,我们的结果还表明 \texttt{linux-ext4} 子系统也具有较高的 Rust 化优先级。鉴于 VFS 安全抽象已被提出\cite{56_rfl_vfs_safety_2024},且已有多个基于 Rust 的文件系统实现\cite{32_puzzlefs_buiild_vfs_2023,36_tarfs_buiild_vfs_2023},我们预期 RFL 接下来的扩展将集中在 \texttt{ext4} 文件子系统上,并期待它能有效提升内存/线程安全保障能力。

\begin{figure}[htbp]
  \centering
  \includegraphics[width=0.85\linewidth]{subsystem_analysis.jpg}
  \caption{Linux 子系统分析结果。每个点代表一个子系统,点的大小表示其代码体量,相对颜色表示其 Rust 化的紧迫程度(蓝色:最紧迫,红色:最不紧迫)}
  \label{fig:subsystem-priority}
\end{figure}

    \section{相关工作}

\subsection*{对野外环境中 Rust 使用的理解}

已有研究从多个角度分析了 Rust 在实际项目中的 unsafe 使用情况。例如,部分工作以 Tock\cite{94_amit_levy_multiprogramming_2017}、TiKV\cite{6_tikv_transactional_key_2016} 与 Redox\cite{5_redox_unix_like_2015} 等流行 Rust 项目为对象,研究其 unsafe 使用场景及背后的设计动因\cite{71_vytautas_astrauskas_how_2020,79_ana_nora_evans_2020,83_sandra_höltervennhoff_wouldnt_2023,111_xiaoye_zheng_closer_2023},从而为 Rust 开发者提供如何更合理使用 unsafe 的实践指导。

另一些研究则聚焦于用户程序和 Rust 编译器中的 bug,分析其底层原因\cite{103_boqin_qin_understanding_2020,107_xinmeng_xia_understanding_2023,108_hui_xu_memory_safety_2021}。与这些工作不同,本文专注于对 Rust-for-Linux (RFL) 项目的研究,从 Rust 语言与 Linux 内核代码库的融合过程中提炼关键经验与教训,并评估 RFL 是否达到了其最初设定的目标,如提升安全性与实现零开销抽象。

\subsection*{Rust 的运行时开销}

尽管 Rust 旨在达到与 C 相当的运行效率,学界仍有不少研究探索其在实际中的开销表现\cite{73_abhiram_balasubramanian_system_2017,81_amélie_gonzalez_takeaways_2023,110_yuchen_zhang_towards_2023}。这类研究通常将同一程序用 Rust 和 C 各自实现并比较其性能差异,但使用的基准程序大多较为简单,例如不足百行代码的经典算法示例,难以反映操作系统等复杂程序的真实情况。

Gonzalez 等\cite{81_amélie_gonzalez_takeaways_2023} 实现了一个非平凡的 UDP 驱动,并从端到端视角比较 Rust 与 C 实现的差异。而我们则基于 Linux 驱动,系统性地度量了运行时指标与对象体积开销,从而更全面地评估 Rust 的开销表现。

\subsection*{基于 Rust 的操作系统内核}

近年来,Rust 已成为系统软件开发的热门语言,学术界与工业界均探索基于 Rust 开发新型操作系统与内核,以适配嵌入式设备与个人计算机等不同场景\cite{73_abhiram_balasubramanian_system_2017,92_stefan_lankes_exploring_2019,93_amit_levy_ownership_2015,94_amit_levy_multiprogramming_2017,95_amit_levy_case_2017,99_a_light_reenix_2015,101_vikram_narayanan_redleaf_2020}。

然而,这些工作大多关注如何构建纯 Rust 内核,并未深入探讨 Rust 与现有 C 代码融合的过程,而这正是当前面临的技术挑战。同时,已有研究亦未关注如何构建安全抽象以重用遗留代码。因此,本文所获得的经验与结论对于未来的 Rust 内核开发也具有重要参考价值。

\subsection*{Rust 的增强机制}

为使 Rust 更安全且更易于开发者使用,已有大量研究提出相关增强机制。一部分工作关注影响 Rust 普及率的因素,并提出降低学习门槛的建议\cite{77_michael_coblenz_multimodal_2020,112_shuofei_zhu_learning_2022}。另一些工作则致力于提升 unsafe 块的安全性,手段包括形式化验证\cite{86_ralf_jung_rustbelt_2017}、静态分析\cite{98_zhuohua_li_mirchecker_2021}、沙箱机制\cite{72_yechan_bae_rudra_2021,74_inyoung_bang_trust_2023,89_paul_kirth_pkru_2022,91_benjamin_lamowski_sandcrust_2017,100_peiming_liu_securing_2020,104_elijah_rivera_keeping_2021} 等。

我们的研究同样为更好地将 Rust 融入 Linux 提供了启示,包括如何处理 unsafe 安全性、如何设计测试机制以及如何引导社区共建等方面。

    \section{结论}

本文对 Rust-for-Linux(RFL)项目进行了系统性深入研究。RFL 是首个致力于将 Rust 引入 Linux 内核以增强其安全性与可维护性的开源项目,并日益受到社区关注与采纳。

我们首先从安全抽象层的构建与 Rust 驱动的实现两个维度出发,剖析了 RFL 当前的实现现状,揭示了 Rust 语言特性与传统内核编程范式之间所存在的张力。接着,我们分析了 RFL 是否以及在多大程度上兑现了其“构建更安全且零开销内核”的承诺。研究结果表明,RFL 在提升安全性方面确实有所贡献,但仍存在无法由编译器察觉的隐蔽性缺陷;而由 Rust 与 Linux 内核语义不兼容所引发的性能开销亦难以彻底避免。

最后,本文总结了研究过程中的关键经验与教训,希望为今后 RFL 的持续演进与开发提供有价值的指导与借鉴。

    
    % 书面翻译的参考文献
    % 默认使用正文的参考文献样式;
    % 如果使用 BibTeX,可以切换为其他兼容 natbib 的 BibTeX 样式。
    \bibliographystyle{unsrtnat}
    % \bibliographystyle{IEEEtranN}
    
    % 默认使用正文的参考文献 .bib 数据库;
    % 如果使用 BibTeX,可以改为指定数据库,如 \bibliography{ref/refs}。
    \printbibliography
    
    \appendix
    \section{案例研究:内核空间中 Rust 编程的灵活性问题}

图~\ref{fig:rust-inflexibility} 展示了一个典型案例,说明 Rust 的安全规则在某些场景下限制了内核编程的灵活性\cite{96_hongyu_li_rros_2023}。该例实现了一个字符设备的动态数组,左侧为内核 C 版本,右侧为 Rust 版本,其结构定义如图~\ref{fig:rust-inflexibility}(a) 所示。

虽然两者结构上相似,Rust 实现存在关键缺陷:用于指定工厂数组大小的常量 $N$ 一旦设定即不可更改(例如设置为 256),这一不可变常量将固定应用于所有工厂实例。无论线程工厂可能需要 256 个元素,代理工厂仅需 8 个元素,Rust 下二者都只能选择统一大小(要么都为 256,要么都为 8),从而造成内存浪费或碎片。

相比之下,C 内核实现则可通过指针修改 \texttt{len} 字段,从而轻松实现字符设备的动态注册与数组扩展。

尽管如此,开发者可借助一系列绕过方案克服上述限制,如图~\ref{fig:rust-inflexibility}(b) 所示。首先需手动定义 \texttt{dyn\_num} trait 并声明其使用(如 \texttt{use\_elements} 函数),并通过 \texttt{dyn trait} 机制启用动态分发。接着,引入常量泛型参数 $T$ 用以表达每个实例的不同元素数量。最后,为具体类型实现该 trait,以分别定义线程工厂(256)与代理工厂(8)的大小。

上述方案虽然可行,但会增加开发复杂度,并引入运行时检查,从而影响性能。

\begin{figure}[htbp]
  \centering
  \includegraphics[width=0.85\linewidth]{rust_inflexibility_example.jpg}
  \caption{示例展示了 RFL 驱动编写中 Rust 的灵活性问题}
  \label{fig:rust-inflexibility}
\end{figure}

\section{带调试信息的驱动体积开销}

在调试阶段,开发者可能需要保留 Rust 驱动的调试信息。图~\ref{fig:debug-info-size} 展示了带调试信息(如 \texttt{debug\_foo} 区段)下 Rust 与 C 驱动体积的对比。

即使是功能未完全实现的 Rust 驱动,其二进制体积也比对应的 C 实现大 3.9--6.6 倍。主要原因在于 Rust 广泛使用泛型编程,导致生成了更多符号及更长的符号名,从而显著扩大调试段的大小。

在资源受限的嵌入式设备中(如闪存和内存仅数 MB 级别\cite{102_pierre_olivier_flashmon_2014}),这种开销是不可忽视的。因此,为实现调试功能,如何压缩驱动体积是嵌入式系统开发中的关键挑战\cite{87_asim_kadav_understanding_2012}。

\begin{figure}[htbp]
  \centering
  \includegraphics[width=0.8\linewidth]{debug_info_comparison.jpg}
  \caption{Rust 与 C 驱动在启用调试段下的体积对比。带 * 的表示 Rust 驱动未实现完整功能}
  \label{fig:debug-info-size}
\end{figure}

\section{内核社区对 Rust 的看法}

为更深入了解 Linux 内核社区对 Rust 的态度,我们收集了截至 2023 年 8 月 5 日,在 LWN 与 YCombinator 平台上关于 Rust 驱动开发的相关帖子,并利用 ChatGPT 对其进行分析。

分析分为两个部分:一是情感分析(sentiment analysis),将观点划分为正向与负向;二是意见挖掘(opinion mining),识别正负观点背后的具体原因。分析结果如图~\ref{fig:community-opinion-rust} 所示。

总体来看,Rust 因其安全性与性能获得社区较高认可。然而,开发者最关心的问题仍是 Rust 较高的学习曲线。这表明,Rust 要在内核领域获得更广泛接受,还需时间逐步验证其稳定性与实用性。

\begin{figure}[htbp]
  \centering
  \includegraphics[width=0.85\linewidth]{rust_community_opinion.jpg}
  \caption{开发者对 RFL 在 Linux 中使用的观点统计与分析}
  \label{fig:community-opinion-rust}
\end{figure}

    
    % 书面翻译对应的原文索引
    % \begin{translation-index}
    %   \nocite{mellinger1996laser}
    %   \nocite{bixon1996dynamics}
    %   \nocite{carlson1981two}
    %   \bibliographystyle{unsrtnat}
    %   \printbibliography
    % \end{translation-index}
    
\end{translation}
    