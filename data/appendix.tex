\chapter{附录:参考文献整理}

本附录汇总了论文编写过程中参考和引用的关键资源,包括开源项目、官方文档、学术论文、技术报告等,作为论文主体内容的补充。

\section{项目与文档资源}

\begin{enumerate}
  \item ArceOS 官方 GitHub 仓库:\url{https://github.com/arceos-org}
  \item Starry-next 操作系统 GitHub 仓库:\url{https://github.com/arceos-org/starry-next}
  \item ArceOS Tutorial Book:\url{https://rcore-os.cn/arceos-tutorial-book/}
  \item ArceOS 官方网站:\url{http://arceos.org}
  \item RCore OS Discussion:\url{https://github.com/orgs/rcore-os/discussions/21}
  \item Stellarium 天文软件:\url{https://stellarium.org/}
  \item HelenOS 微内核操作系统:\url{https://en.wikipedia.org/wiki/HelenOS}
  \item SpaceNetLab: StarryNet 框架:\url{https://github.com/spacenetlab/starrynet}
\end{enumerate}

\section{技术报告与实验论文}

\begin{enumerate}[resume]
  \item Liang, Y. (2023). \textit{ArceOS 文件系统架构设计与实现}. OS Lab Report.
  \item Gabber, E., Small, C., Bruno, J., Brustoloni, J., \& Silberschatz, A. (2024). \textit{The Pebble Component-Based Operating System}. ResearchGate.
  \item Lawall, J. (2022). \textit{THINK: A Software Framework for Component-based Operating Systems}. Inria.
  \item Barkhausen Institut. (2024). \textit{Composable Operating System}. Barkhausen Institut.
  \item Feske, N. (2024). \textit{Genode Foundations}. Genode Labs.
  \item Vaishnav, A., Pham, K. D., Powell, J., \& Koch, D. (2020). \textit{FOS: A Modular FPGA Operating System for Dynamic Workloads}. arXiv:2001.09990.
  \item Ababneh, E., Al-Ali, Z., Ha, S., Han, R., \& Keller, E. (2018). \textit{Elasticizing Linux via Joint Disaggregation of Memory and Computation}. arXiv:1806.00885.
  \item Kristiansson, J. (2024). \textit{ColonyOS -- A Meta-Operating System for Distributed Computing Across Heterogeneous Platforms}. arXiv:2403.16486.
  \item Hu, J., Lu, E., Holland, D. A., Kawaguchi, M., Chong, S., \& Seltzer, M. I. (2022). \textit{Towards Porting Operating Systems with Program Synthesis}. arXiv:2204.07167.
  \item Synopsys. (2024). \textit{ARC Operating Systems}. Synopsys.
\end{enumerate}