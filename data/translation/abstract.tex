\section{摘要}
    
Linux 发展已逾三十年,早已成为现代数字世界的计算基础:无论是庞大复杂的主机(如超级计算机),还是廉价轻量的嵌入式设备(如物联网设备),无数应用都构建于其之上。然而,这一基础设施自诞生之日起便一直受到内存与并发错误的困扰,其根源在于 C 语言对内存操作的宽松约束,允许大量“野蛮”的访问行为。

近期项目 Rust-for-Linux(RFL)有望一劳永逸地缓解 Linux 内核中的安全问题。RFL 通过将 Rust 的静态所有权机制与类型检查引入内核代码,使得内核在不牺牲性能的前提下,有可能摆脱内存与并发错误的困扰。尽管 RFL 正在逐步成熟并被并入 Linux 主线内核,其安全性与性能是否能够真正兼得仍未得到系统性研究。

为此,本文开展了对 RFL 的首次实证研究,旨在揭示其发展现状与带来的益处,特别关注 Rust 如何与 Linux 内核融合,以及该融合是否能在不增加系统开销的前提下保障驱动程序的安全性。我们收集并分析了六个关键的 RFL 驱动,涵盖数百个 issue 与 PR、数千次 GitHub 提交与邮件往来,以及超 1.2 万条 Zulip 论坛讨论。

研究发现:虽然 Rust 在一定程度上缓解了内核中的安全漏洞,但其能力仍不足以彻底根除此类问题。更重要的是,如果使用不当,Rust 所提供的安全保障反而可能带来显著的运行时开销与开发负担。
