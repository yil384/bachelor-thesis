\section{RQ3:有哪些经验与教训?}

本节总结了我们在本次实证研究中获得的关键经验与教训。

\subsection*{面向 RFL 开发者的经验教训}

为提升 RFL 及其 Rust 驱动的安全性与实用性,开发者可考虑如下建议:

\begin{enumerate}
  \item \textbf{不要将 Rust 内建的安全检查器视为万能工具}。正如第 \ref{sec:security-analysis} 节所示,Rust 的安全检查器在真实驱动中无法识别某些语义级 bug。因此,建议开发者结合更强大的分析工具(如 RustBelt\cite{86_ralf_jung_rustbelt_2017} 与 miri\cite{54_interpreter_rust_mid_level_2024})以弥补 Rust 内建机制的不足。

  \item \textbf{从“所有权”视角构建安全的内核抽象,并管理驱动资源}。这与传统内核编程中以“内存”视角出发的方式不同。编程前应先明确结构体间的所有权关系,否则在使用智能指针错误管理所有权后,修复代价将极为高昂。

  \item \textbf{接受 unsafe 作为最后的选项}。Rust 的安全规则在处理内核内存操作时,常需大量使用智能指针与泛型编程,导致内存开销巨大(详见第 \ref{sec:performance-analysis} 节)。此时,如若经过充分审查,可选择使用 unsafe 实现以换取实用性。
\end{enumerate}

\subsection*{面向 RFL 社区的建议:如何扩展 RFL 的应用范围}

决定将 RFL 优先应用于哪些未来内核驱动或子系统至关重要,因为这需要大量的开发资源与长期维护承诺。为此,我们构建了一个效益模型,用于评估“Rust 化(Rustify)”的优先级。

该效益定义为:某驱动或子系统中可通过 Rust 机制修复的累计 bug/漏洞数量与其代码规模(LoC)的比值。直观理解是:一个规模较小但含有大量内存/线程类 bug 的子系统,其 Rust 化的价值更高。这一思路已得到社区共识的支持\cite{35_rfl_driver_selection_2023}。

为量化上述指标,我们遍历了各驱动的 git 历史,收集其至今出现的所有安全问题,并手动审查其中已修复的 bug,筛选出那些可通过 Rust 安全机制避免的内存/线程类 bug(依据第 \ref{sec:rust-safety-mechanism} 节定义)。我们共分析了来自 79 个不同子系统的 2500 多个驱动,并将分析结果绘制于图~\ref{fig:subsystem-priority}。

平均而言,每个子系统包含 1.3 个 bug/每千行代码;但在不同子系统间,该比值差异显著。其中,\texttt{linux-block} 子系统由于每行代码含最多 bug(共 438 个修复中包括 113 个数据竞争 bug 与 98 个悬垂指针 bug),因此被我们建议优先 Rust 化。令人欣喜的是,该子系统的 \texttt{null block} 驱动已经被社区用 RFL 重写,并在第 \ref{sec:performance-analysis} 节进行了测试验证。

除 \texttt{linux-block} 外,我们的结果还表明 \texttt{linux-ext4} 子系统也具有较高的 Rust 化优先级。鉴于 VFS 安全抽象已被提出\cite{56_rfl_vfs_safety_2024},且已有多个基于 Rust 的文件系统实现\cite{32_puzzlefs_buiild_vfs_2023,36_tarfs_buiild_vfs_2023},我们预期 RFL 接下来的扩展将集中在 \texttt{ext4} 文件子系统上,并期待它能有效提升内存/线程安全保障能力。

\begin{figure}[htbp]
  \centering
  \includegraphics[width=0.85\linewidth]{subsystem_analysis.jpg}
  \caption{Linux 子系统分析结果。每个点代表一个子系统,点的大小表示其代码体量,相对颜色表示其 Rust 化的紧迫程度(蓝色:最紧迫,红色:最不紧迫)}
  \label{fig:subsystem-priority}
\end{figure}
