\section{结论}

本文对 Rust-for-Linux(RFL)项目进行了系统性深入研究。RFL 是首个致力于将 Rust 引入 Linux 内核以增强其安全性与可维护性的开源项目,并日益受到社区关注与采纳。

我们首先从安全抽象层的构建与 Rust 驱动的实现两个维度出发,剖析了 RFL 当前的实现现状,揭示了 Rust 语言特性与传统内核编程范式之间所存在的张力。接着,我们分析了 RFL 是否以及在多大程度上兑现了其“构建更安全且零开销内核”的承诺。研究结果表明,RFL 在提升安全性方面确实有所贡献,但仍存在无法由编译器察觉的隐蔽性缺陷;而由 Rust 与 Linux 内核语义不兼容所引发的性能开销亦难以彻底避免。

最后,本文总结了研究过程中的关键经验与教训,希望为今后 RFL 的持续演进与开发提供有价值的指导与借鉴。
