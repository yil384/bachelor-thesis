% !TeX root = ../thuthesis-example.tex

% 中英文摘要和关键字

\begin{abstract}
  随着互联网技术的迅猛发展,网络协议栈在现代操作系统中扮演着越来越重要的角色。轻量级协议栈如 lwIP 和 smoltcp 已在嵌入式系统、物联网设备和资源受限环境中广泛应用。本论文研究并实现了在 ArceOS 操作系统基础上,兼容多架构的网络协议栈接口模块,以支持 lwIP 和 smoltcp 协议栈的集成。通过系统调用接口的设计和实现,提供了统一的网络操作接口,使得操作系统能够灵活切换不同的协议栈,从而满足不同应用场景的需求。

  论文首先分析了 ArceOS 操作系统的设计架构,重点解决了网络协议栈与内存管理、系统调用接口的兼容性问题。基于 Rust 语言的安全性和高性能特性,本文设计了可扩展的网络系统调用接口,并实现了多种网络协议栈的并行支持。通过引入条件编译和特性标志,系统能够根据需求选择集成 lwIP 或 smoltcp 协议栈。为确保协议栈在多核环境中的线程安全,采用了互斥锁和其他同步机制。

  在实现过程中,本论文设计并实现了与 Starry-Next 操作系统架构兼容的网络协议栈接口,通过四种架构实现了常见的网络系统调用,如 SOCKET、BIND、CONNECT、SENDTO、RECVFROM 等,支持多种网络协议栈的动态切换和无缝集成。实验结果表明,该设计能够显著提升网络通信的灵活性、可扩展性和性能,特别是在资源受限的嵌入式设备和高并发网络环境中。

  通过在 QEMU 虚拟平台上进行功能和性能测试,本论文验证了所提出的网络协议栈接口设计的可行性。实验结果表明,该设计不仅能够提高操作系统的网络管理能力,还能提供更高效的网络协议栈支持,满足多样化应用的需求。

  % 关键词用“英文逗号”分隔,输出时会自动处理为正确的分隔符
  \thusetup{
    keywords = {组件化操作系统, 宏内核, ArceOS, starry-next, 系统调用, 轻量级协议栈, lwIP, smoltcp, 内存安全, 并发处理},
  }
\end{abstract}

\begin{abstract*}
  With the rapid development of internet technologies, the network protocol stack plays an increasingly vital role in modern operating systems. Lightweight stacks such as lwIP and smoltcp have been widely adopted in embedded systems, IoT devices, and resource-constrained environments. This paper investigates and implements a cross-architecture network protocol stack interface module based on the ArceOS operating system, enabling the integration of both lwIP and smoltcp. By designing and implementing a unified system call interface, the operating system can flexibly switch between different protocol stacks to meet the demands of various application scenarios.
  
  The paper first analyzes the architectural design of ArceOS, focusing on addressing compatibility issues between network protocol stacks, memory management, and system call interfaces. Leveraging the safety and performance features of the Rust programming language, we design an extensible system call interface that supports multiple protocol stacks in parallel. Through conditional compilation and feature flags, the system can selectively integrate either lwIP or smoltcp. To ensure thread safety in multicore environments, mutexes and other synchronization mechanisms are employed.
  
  During implementation, a network stack interface compatible with the Starry-Next operating system architecture is developed, supporting common system calls such as SOCKET, BIND, CONNECT, SENDTO, and RECVFROM across four architectures. The design enables dynamic switching and seamless integration of multiple network stacks. Experimental results demonstrate that this approach significantly improves flexibility, scalability, and performance of network communication, particularly in resource-limited embedded devices and high-concurrency environments.
  
  Functionality and performance tests conducted on the QEMU virtual platform validate the feasibility of the proposed network stack interface design. This design not only enhances the network management capabilities of the operating system but also provides efficient support for lightweight protocol stacks, meeting the diverse needs of modern applications.
  
  \thusetup{
  keywords* = {Componentized Operating System, Monolithic Kernel, ArceOS, starry-next, System Call, Lightweight Protocol Stack, lwIP, smoltcp, Memory Safety, Concurrent Processing},
  }
\end{abstract*}