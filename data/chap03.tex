\chapter{lwIP 与 smoltcp 网络协议栈的集成与分析}

\section{lwIP 与 smoltcp 协议栈背景与设计理念}
在操作系统的网络子系统中,TCP/IP 协议栈扮演着核心角色。随着嵌入式设备、物联网和轻量级虚拟化的发展,传统全功能协议栈(如 Linux 内核网络栈)面临资源开销大、适配复杂等问题。因此,轻量化、模块化、高可移植性的 TCP/IP 协议栈成为操作系统设计中的关键组件。

在本论文所设计的 ArceOS 与 starry-next 系统中,出于灵活适配多种部署环境的需求,选取了两种轻量级协议栈:\textbf{lwIP} 与 \textbf{smoltcp}。二者在设计理念、目标平台、功能覆盖与性能特点上各具特色,满足从边缘节点到高性能虚拟机等多样化使用场景。

\textbf{lwIP 协议栈简介} \par
lwIP(Lightweight IP)由 Adam Dunkels 于 2001 年提出,最初用于 uIP 项目的高性能补充版本,后由瑞典计算机科学研究院维护。它采用 C 语言实现,结构紧凑,功能全面,适用于具有操作系统支持或裸机环境的嵌入式平台。

\paragraph{设计目标} 其核心目标是提供完整 TCP/IP 协议支持,最小化资源消耗,同时具备良好的可移植性。

\paragraph{主要特性} 
\begin{itemize}
  \item 支持 TCP、UDP、ICMP、IGMP、IPv4/IPv6 协议;
  \item 支持多线程/无线程运行模式(通过 NO\_SYS 配置切换);
  \item 具备 DNS、DHCP、SNMP、PPP 等高级协议模块;
  \item 内存使用灵活:支持内存池、堆、pbuf 链等多种分配方式。
\end{itemize}

\paragraph{适用场景} 
\begin{itemize}
  \item 嵌入式系统与中型设备,运行 RTOS 或裸机主循环;
  \item 对网络兼容性要求高,需要支持 DNS、DHCP 等协议;
  \item 高吞吐、高并发的微服务器部署(如边缘网关)。
\end{itemize}

\textbf{smoltcp 协议栈简介} \par
smoltcp 是一个由 Rust 社区开发、面向 IoT 和嵌入式平台的现代化网络协议栈。以安全、最小化和可组合为核心设计原则,其源码结构清晰,便于阅读和集成,尤其适合资源极端受限的系统。

\paragraph{设计目标} 实现高度模块化、零动态内存分配的 TCP/IP 协议栈,适应异步、事件驱动的系统架构。

\paragraph{主要特性}
\begin{itemize}
  \item 完全采用 Rust 编写,利用语言级内存安全机制;
  \item 支持 TCP、UDP、ICMPv4、ARP、IPv4 等协议;
  \item 零动态内存分配:所有资源在初始化阶段分配;
  \item 提供非阻塞 API,适用于 async/await 与轮询模型。
\end{itemize}

\paragraph{适用场景}
\begin{itemize}
  \item 极度资源受限设备,如 IoT 控制器、传感器节点;
  \item 使用 Rust 异步模型的裸机或嵌入式系统;
  \item 需要语言级内存安全保证的嵌入式研发环境。
\end{itemize}

\begin{figure}
  \centering
  \subcaptionbox{lwIP 协议栈架构示意图 \label{fig:lwip-arch}}
    {\includegraphics[width=0.47\linewidth]{figures/lwip_arch.pdf}}
  \subcaptionbox{smoltcp 协议栈架构示意图 \label{fig:smoltcp-arch}}
    {\includegraphics[width=0.47\linewidth]{figures/smoltcp_arch.pdf}}
  \caption{lwIP 与 smoltcp 协议栈架构对比}
  \label{fig:protocol_stack_arch}
\end{figure}

\section{功能与架构对比分析}

\begin{table}[h]
  \centering
  \caption{lwIP 与 smoltcp 核心架构比较}
  \begin{tabular}{p{4cm}p{5cm}p{5cm}}
    \toprule
    \textbf{特性维度} & \textbf{lwIP} & \textbf{smoltcp} \\
    \midrule
    编程语言 & C & Rust \\
    内存管理 & 动态/静态混合,支持堆与内存池 & 完全静态,零动态分配 \\
    协议覆盖 & TCP/UDP/ICMP/IGMP 等 & TCP/UDP/ICMP/ARP 等 \\
    多线程支持 & 支持(sys layer) & 非阻塞设计,支持 async \\
    定时机制 & 内置定时器 + 任务调度 & 用户提供 tick 驱动 \\
    使用场景 & 嵌入式、软路由、RTOS 环境 & IoT、嵌入式裸机 \\
    安全特性 & 无语言级保护,依赖开发者控制 & Rust 保证内存安全 \\
    可裁剪性 & 配置选项众多,可模块裁剪 & 特性标志清晰,易于组合 \\
    文档支持 & 完善的移植与集成文档 & Rust 社区持续维护,结构规范 \\
    \bottomrule
  \end{tabular}
  \label{tab:arch_compare}
\end{table}

lwIP 使用传统网络协议栈的同步模型,数据包从驱动上传,经过 pbuf 缓存池分发到各协议模块,最后到达应用 socket 层。smoltcp 使用事件轮询机制,应用主动 poll() 网络接口,读取缓存区或传递发包数据,由应用控制时序。

\section{协议栈启动与数据处理流程}

为了进一步理解 lwIP 与 smoltcp 在 ArceOS 中的运行机制与设计思路,本节对两种协议栈在系统启动、网络设备初始化、数据包处理及与应用交互过程中的完整流程进行了深入分析,并结合流程图进行直观展示。这不仅有助于理解其内在运行逻辑,也为后续调试与性能优化提供技术基础。

\textbf{lwIP 启动与数据处理流程} \par

lwIP 作为一款成熟的轻量级网络协议栈,其启动过程主要包括协议栈结构初始化、网络接口注册、回调函数绑定以及定时机制的启用等步骤。由于 lwIP 支持两种运行模式(带操作系统或裸机模式),在 ArceOS 的集成中我们采用的是裸机模式,即 \texttt{NO\_SYS=1} 配置下的手动轮询调度方式。

协议栈启动后的数据处理流程可概括如下:

\begin{itemize}
  \item 在系统初始化阶段调用 \texttt{lwip\_init()} 函数完成协议栈的全局初始化,包括内存池、pbuf 缓冲区、TCP 控制块(TCB)等资源的配置;
  \item 使用 \texttt{netif\_add()} 函数注册网络接口,并指定输入处理回调函数(如 \texttt{ethernet\_input}),同时通过 \texttt{netif\_set\_up()} 启用该接口;
  \item 对于定时事件,如 TCP 重传、ARP 超时等,使用定时任务线程,来进行定期调用 \texttt{sys\_check\_timeouts()} ,从而驱动协议栈内部的定时器;
  \item 网络驱动在接收到数据帧后,将其封装为 \texttt{pbuf} 结构,并通过输入回调函数传递至协议栈入口;
  \item 协议栈根据数据类型(IP、ARP、ICMP等)进行解析,并分发到相应的协议模块;
  \item 应用程序通过 socket 或 netconn 接口与 TCP 或 UDP 层交互,完成数据的发送与接收。
\end{itemize}

\textbf{smoltcp 启动与数据处理流程} \par

与 lwIP 不同,smoltcp 遵循高度模块化和最小化的设计原则,采用主动式轮询模型,并不依赖于中断或回调机制。这使得其在嵌入式或裸机系统中具备更好的时序控制能力,适合资源紧张或任务驱动型场景。

其启动与数据处理流程如下:

\begin{itemize}
  \item 使用 \texttt{InterfaceBuilder} 创建网络接口结构体,配置本地 IP、MAC 地址、缓冲池、MTU 等信息;
  \item 使用 \texttt{SocketSet} 管理全部 socket 资源,每个 socket 均以静态方式注册;
  \item 通过周期性调用 \texttt{iface.poll()} 函数轮询网络设备,读取驱动中接收到的数据包,并触发协议层逻辑处理;
  \item poll 函数内部根据 socket 状态进行数据转发、连接维护、ACK 回应等操作;
  \item 应用程序使用非阻塞方式操作 socket,主动轮询读取或写入数据;
  \item 所有资源均在初始化时静态分配,运行过程中不进行动态内存分配,提升系统可预测性与实时性。
\end{itemize}

图 \ref{fig:network_start_flow} 展示了 lwIP 与 smoltcp 在 ArceOS 中的启动与数据处理流程对比。两种协议栈在设计理念、资源管理、数据处理机制等方面存在显著差异,适用于不同的应用场景。

\begin{figure}[htbp]
  \centering
  \includegraphics[width=0.5\textwidth]{figures/net_start_flow.pdf}
  \caption{ArceOS 网络协议栈启动流程比较图}
  \label{fig:network_start_flow}
\end{figure}

两种协议栈在启动流程、资源配置和数据处理机制上存在显著差异:

\begin{itemize}
  \item \textbf{调度模式:} lwIP 依赖定时器驱动和回调函数,适用于具备定时机制或 RTOS 支持的系统;而 smoltcp 全部采用显式调用 \texttt{poll()},适合裸机与 async 模型;
  \item \textbf{资源管理:} lwIP 使用动态和静态混合方式分配内存,而 smoltcp 完全避免动态内存分配,更适合对实时性要求高的场合;
  \item \textbf{使用方式:} lwIP 适合传统 socket 模型,smoltcp 则倾向于事件驱动架构,接口清晰但灵活度略低;
  \item \textbf{调试与扩展性:} lwIP 功能更丰富,但内部模块耦合较高;smoltcp 结构简洁,模块之间边界清晰,便于移植与扩展。
\end{itemize}

两种协议栈各有侧重,在 ArceOS 的网络系统中实现双协议栈支持,能够覆盖从轻量 IoT 到高负载边缘网关的多种使用场景,为系统的灵活部署和协议栈演进提供了坚实基础。

为了增强 ArceOS 在网络通信方面的性能与兼容性,系统同时集成了两种轻量级协议栈:Rust 实现的 \texttt{smoltcp} 与 C 实现的 \texttt{lwIP}。两者均通过统一的接口抽象在 \texttt{axnet} 中封装,对上层应用提供一致的 Socket API,包括对 TCP、UDP、DNS 等常用网络功能的支持。

相较而言,\texttt{lwIP} 在多连接处理、高并发负载与协议功能完备性方面展现出更佳表现,适合需要较强协议兼容性和吞吐性能的应用场景。

\paragraph{协议栈适配机制} 
ArceOS 中通过实现 \texttt{NetDevices} 接口将底层驱动与协议栈解耦,并为 \texttt{lwIP} 提供裸机级初始化流程(\texttt{lwip\_init})、设备注册(\texttt{netif\_add})与数据收发接口封装(\texttt{myif\_input}, \texttt{myif\_link\_output})。该流程兼容 IPv4/IPv6,支持零拷贝内存复用机制,提高数据处理效率。

\paragraph{TCP 支持优化} 
在 TCP 支持方面,\texttt{lwIP} 通过回调方式处理连接与数据交互。适配时,为每个 TCP 连接创建固定地址的内存对象,并通过互斥锁实现阻塞等待功能,实现接口 \texttt{connect}、\texttt{bind}、\texttt{accept}、\texttt{recv}、\texttt{send} 等函数以支持标准 Socket 使用模式。

\paragraph{UDP 与 DNS 支持} 
对于 UDP 协议,\texttt{lwIP} 利用其 Raw API 实现非连接数据报通信,并支持自定义端口监听和目标发送功能。DNS 查询采用非阻塞模式,通过状态回调与缓存命中快速返回,最终提供一个统一的阻塞式地址解析接口。

\paragraph{性能与稳定性对比} 
在标准测试环境下使用 Apache Benchmark 工具对 HTTP 服务并发访问进行测试,结果如下:

\begin{table}
  \centering
  \caption{smoltcp 与 lwIP 在 HTTP 测试下的请求处理性能(单位:RPS)}
  \begin{tabular}{ccc}
  \toprule
  并发数 & smoltcp & lwIP \\
  \midrule
  1  & 7123 & 7726 \\
  2  & 7444 & 8698 \\
  5  & 7747 & 8752 \\
  10 & 7820 & 8737 \\
  50 & 7406 & 8704 \\
  100& 7480 & 8764 \\
  \bottomrule
  \end{tabular}
\end{table}

由上可见,在多种并发情况下,lwIP 性能均优于 smoltcp,尤其在高并发下保持更强稳定性,且请求失败率更低,不易卡顿。

\paragraph{优化措施} 
为进一步提高 lwIP 性能,系统引入以下优化:
\begin{itemize}
  \item \textbf{收包零拷贝}:利用 \texttt{pbuf\_custom} 配合自定义析构函数,实现驱动层内存复用,避免重复拷贝。
  \item \textbf{内存池优化}:对 TCP 连接、数据结构、队列等进行预分配或内存池管理,显著降低堆分配频率。
  \item \textbf{参数调优}:调整 \texttt{TCP\_MSS}、\texttt{TCP\_WND}、\texttt{MEMP\_NUM\_TCP\_PCB} 等参数,提升窗口大小和连接并发能力。
  \item \textbf{任务处理机制}:基于裸机模式实现的轮询与超时检测逻辑结合 \texttt{lwip\_loop\_once} 调度函数,增强协议栈响应性。
\end{itemize}

\section{lwIP 与 smoltcp 的实现机制与代码分析}

\subsection{TCP Socket 实现机制对比}

\textbf{1. 连接与状态管理机制}

lwIP 和 smoltcp 在 TCP 状态机处理方面有本质区别:lwIP 主要通过 \\C 语言的全局状态变量与回调注册处理连接建立和数据传输;smoltcp 则采用 Rust 的状态封装与安全抽象,通过状态枚举 \texttt{State::SynSent}, \texttt{State::Established} 等来判定连接状态。其连接建立过程使用接口:

\begin{itemize}
  \item lwIP: \texttt{tcp\_connect(pcb, ip, port, connect\_callback)} 注册回调,在异步事件中设定状态;
  \item smoltcp: 直接使用 \texttt{socket.connect()} 发起连接,并封装连接检查逻辑于 \texttt{poll()} 中。
\end{itemize}

\textbf{2. 非阻塞与线程模型支持}

smoltcp 提供 \texttt{nonblock} 原子布尔量,通过 \texttt{block\_on()} 包装轮询逻辑实现非阻塞 I/O。而 lwIP 则通过 \texttt{NO\_SYS=1} 模式实现轮询机制,并手动管理等待队列。

代码中,smoltcp 通过 \texttt{TcpSocket::poll()} 实现如下轮询:

\begin{listing}[!ht]
  \captionsetup{name=代码}
  \begin{minted}{rust}
  match socket.state() {
      State::SynSent => false, // 等待服务器回应
      State::Established => true, // 已连接
      _ => false,
  }
  \end{minted}
  \caption{smoltcp TCP 状态轮询示例}
  \label{lst:TCPSample}
\end{listing}

而 lwIP 中通过定期调用 \texttt{lwip\_loop\_once()} 推进协议状态。

\subsection{TCP 数据收发与缓冲管理}

\textbf{1. 数据接收缓冲}

在数据接收方面:
\begin{itemize}
  \item smoltcp 使用 \texttt{recv\_slice()} 直接读取内部环形缓冲区,结合 \texttt{may\_recv()} 控制连接关闭状态;
  \item lwIP 接收数据使用 \texttt{recv\_callback} 存入 \texttt{recv\_queue},以 \texttt{pbuf} 结构形式保存,每次拷贝需计算偏移量。
\end{itemize}

如下是 lwIP 的数据读取逻辑:

\begin{listing}[!ht]
  \captionsetup{name=代码}
  \begin{minted}{rust}
  let (p, offset) = recv_queue.pop_front().unwrap();
  let copy_len = min(len - offset, buf.len());
  buf[0..copy_len].copy_from_slice(&payload[offset..offset + copy_len]);
  if offset + copy_len < len {
      recv_queue.push_front((PbuffPointer(p), offset + copy_len));
  } else {
      pbuf_free(p);
  }
  \end{minted}
  \caption{lwIP TCP pbuf 数据读取片段}
  \label{lst:lwIPPbuf}
\end{listing}

\textbf{2. 数据发送路径}

两者在发送逻辑上也存在差异:
\begin{itemize}
  \item lwIP 调用 \texttt{tcp\_write()} + \texttt{tcp\_output()},要求手动控制缓冲区使用与传输输出;
  \item smoltcp 采用 \texttt{send\_slice()} 接口,由内部机制控制发送窗口与 ACK 协议。
\end{itemize}

\subsection{UDP 实现策略对比}

UDP 协议在 smoltcp 中作为轻量模块实现,封装了基本的 \texttt{send\_to} 与 \texttt{recv\_from} 接口,支持连接式与无连接式使用。smoltcp 使用 Rust 中的 \texttt{RwLock<Option<Endpoint>>} 封装本地和远程地址。

lwIP 则通过注册 \texttt{udp\_recv} 回调,将数据以 \texttt{pbuf} 加入队列,类似 TCP 接收机制。

\subsection{设备接口封装与拓展性分析}

\textbf{1. smoltcp 的 Device Trait}

smoltcp 通过 \texttt{Device} trait 设计接口,要求实现 \texttt{receive()}, \texttt{transmit()}, \texttt{capabilities()} 三个方法。例如 loopback 接口如下:

\begin{listing}[!ht]
  \captionsetup{name=代码}
  \begin{minted}{rust}
  impl Device for LoopbackDev {
      fn receive(...) -> Option<(RxToken, TxToken)> {
          self.queue.pop_front().map(...)
      }
      fn transmit(...) -> Option<TxToken> { ... }
      fn capabilities(...) -> DeviceCapabilities { ... }
  }
  \end{minted}
  \caption{smoltcp Loopback 接口结构体实现摘要}
  \label{lst:LoopbackImpl}
\end{listing}

\textbf{2. lwIP 的回调与全局状态模型}

lwIP 所有设备行为(收包、发包)都基于驱动调用 pbuf 提交与调用者调度进行,不支持多实例或 trait 抽象,扩展性受限。

\subsection{lwIP 的内存结构与 pbuf 管理机制}
在 lwIP 中,\texttt{pbuf}(protocol buffer)是网络数据在协议栈中传输的基本结构,起到缓存和抽象封装数据包的作用。pbuf 支持链式组织,以应对碎片化或 MTU 分段等情况。

pbuf 提供如下几种类型:
\begin{itemize}
  \item \textbf{PBUF\_RAM}:在 RAM 中分配的可修改数据块,常用于上层应用构造发出数据;
  \item \textbf{PBUF\_POOL}:来自内存池的 pbuf,适合频繁分配释放的数据包;
  \item \textbf{PBUF\_REF}:仅保存对现有数据的引用,避免内存拷贝;
  \item \textbf{PBUF\_ROM}:用于只读数据(如静态响应内容),仅做引用,不拷贝数据。
\end{itemize}

在 ArceOS 中,pbuf 管理机制通过零拷贝优化得以强化。驱动层接收到网卡数据后,可直接封装为 \texttt{pbuf\_custom},绑定自定义释放函数,从而将原始数据直接传入协议栈,无需额外拷贝。如下所示:

\begin{listing}[!ht]
  \captionsetup{name=代码}
  \begin{minted}{c}
  struct pbuf_custom_custom *pc = alloc_custom_pbuf(buffer, len);
  pc->pbuf.custom_free_function = my_pbuf_free;
  pbuf_chain(p, &pc->pbuf);
  \end{minted}
  \caption{pbuf\_custom 零拷贝封装示例}
  \label{lst:ZeroCopyPbuf}
\end{listing}

该机制配合预分配的 pbuf 池和页对齐分配策略,显著降低内存开销并提升收包性能。

\subsection{lwIP 回调机制与事件驱动流程}
与传统的阻塞 IO 不同,lwIP 在裸机或 NO\_SYS 模式下采用事件驱动模型。其核心机制为对每个 TCP 或 UDP socket 注册回调函数,如 \texttt{tcp\_recv}、\texttt{tcp\_accept} 等。

以下展示了 TCP 收包回调函数在 ArceOS 中的注册与触发:

\begin{listing}[!ht]
  \captionsetup{name=代码}
  \begin{minted}{rust}
  tcp_recv(pcb, Some(recv_callback));
  
  extern "C" fn recv_callback(arg: *mut c_void, pcb: *mut tcp_pcb, p: *mut pbuf, err: err_t) -> err_t {
      if !p.is_null() {
          let socket_inner = unsafe { &mut *(arg as *mut TcpSocketInner) };
          socket_inner.recv_queue.lock().push_back((PbuffPointer(p), 0));
      }
      0
  }
  \end{minted}
  \caption{lwIP TCP 收包回调流程}
  \label{lst:LwipRecvCallback}
\end{listing}

该机制通过预注册的函数实现从驱动中断到用户态事件的快速传递,结合 ArceOS 中断轮询框架,在极简系统中亦能维持 TCP 全功能通信。

\subsection{smoltcp 的 poll 模型与异步适配机制}
smoltcp 的核心为轮询驱动的非阻塞模型。应用需显式调用 \texttt{poll()} 接口进行状态更新与数据交互。在 ArceOS 中,系统通过异步任务调度框架(如 \texttt{axtask::yield\_now})配合轮询操作实现伪阻塞行为:

\begin{listing}[!ht]
  \captionsetup{name=代码}
  \begin{minted}{rust}
  loop {
      SOCKET_SET.poll_interfaces();
      match socket.recv(buf) {
          Ok(len) => return Ok(len),
          Err(AxError::WouldBlock) => yield_now(),
          Err(e) => return Err(e),
      }
  }
  \end{minted}
  \caption{smoltcp 异步 poll 示例}
  \label{lst:SmoltcpPoll}
\end{listing}

此外,smoltcp 支持 \texttt{poll\_connect}、\texttt{poll\_stream} 等状态函数,结合内部状态机对连接状态进行精细化管理,如 TCP 状态迁移、握手确认、接收窗口管理等。

这种设计高度契合 Rust 的异步风格,可进一步封装为 \texttt{Future} 以支持 \texttt{async/await} 语法,提升代码清晰度与并发度。

\textbf{代码实现总结与优化建议} \par

从代码实现层面可以得出如下结论:

\begin{itemize}
  \item smoltcp 在架构上更适合安全抽象和多任务调度;lwIP 则更灵活但对状态一致性管理要求高;
  \item UDP 的收发处理在 smoltcp 中更简洁,但 lwIP 更具扩展性和调试性;
  \item 对于实时或高可靠通信,lwIP 通过强管控缓冲策略提供更稳定性能;
  \item 建议使用 \texttt{SocketInner} 枚举封装两种协议栈,并通过统一 trait 抽象提升系统适配性。
\end{itemize}

下面将进一步分析这两种协议栈在 ArceOS 中如何通过 axnet 模块实现统一集成与适配。

\section{ArceOS 中的协议栈集成与适配实现}

\textbf{统一抽象层设计:axnet 接口} \par
为实现 lwIP 与 smoltcp 的并行使用,ArceOS 定义了一层抽象接口 axnet,封装所有网络设备、地址类型、数据包缓冲区与协议栈交互逻辑。其目标是:
\begin{itemize}
  \item 上层调用透明:应用层不感知底层协议栈实现;
  \item 动态切换协议栈:通过配置切换默认网络协议栈;
  \item 支持多实例:为多租户系统支持多个网络堆栈实例。
\end{itemize}

\textbf{lwIP 的封装适配要点} \par
\begin{itemize}
  \item 提供 socket 接口适配:将 lwIP 的 netconn/socket API 封装为 Rust trait 接口;
  \item 实现非阻塞接口:通过 select/poll 实现 pseudo async 支持;
  \item 启用 IPv6、DHCP、DNS、ICMPv6 模块;
  \item 使用独立线程轮询 lwIP 时间器(NO\_SYS=1 模式)。
\end{itemize}

\textbf{smoltcp 的集成要点} \par
\begin{itemize}
  \item 通过 trait \texttt{Device} 抽象 ArceOS 网络设备结构体;
  \item 将 socket 数据结构 statically 注册于 Interface;
  \item 使用 \texttt{phy::RawSocket} 模拟真实网卡接口;
  \item 调用 \texttt{poll()} 方法驱动状态机。
\end{itemize}

由于 lwIP 是一款应用广泛的轻量级 TCP/IP 协议栈,拥有完整的 TCP、UDP、ICMP、ARP 等协议支持,适用于嵌入式和资源受限环境。其设计关注内存效率和可扩展性,是多数微型操作系统默认选项。

在 ArceOS 中,我们对 lwIP 进行了如下集成和优化:
\begin{itemize}
    \item \textbf{API 封装与接口统一}:将 lwIP 的 socket 接口封装在统一的 Rust 接口之下,对外暴露标准的 POSIX 风格操作,使上层无需感知底层实现差异;
    \item \textbf{协议栈模块优化}:对 lwIP 的内存分配策略进行分析,优先使用内存池进行缓冲区分配;收包采用 pbuf 引用零拷贝机制,减少内存拷贝;
    \item \textbf{裸机移植支持}:开启 \texttt{NO\_SYS} 模式,在无 RTOS 情况下通过主循环调度 lwIP 定时器,兼容 ArceOS 的无线程模型;
    \item \textbf{IPv6 与 DNS 支持}:通过配置 \texttt{lwipopts.h} 开启 IPv6 栈及 DNS 客户端功能,满足现代网络栈对多协议支持的需求;
    \item \textbf{调优配置参数}:对 TCP 窗口大小、最大报文段 MSS、连接池数量等参数进行调优,提升高并发环境下协议栈性能。
\end{itemize}

性能测试表明,在 Apache Benchmark 压力下,lwIP 相较 smoltcp 拥有更高的并发处理能力,在 100 并发连接下 RPS 提升达 \textasciitilde12\%。

由于 smoltcp 是专为嵌入式系统和 IoT 场景设计的轻量协议栈。其代码简洁、内存需求极低,且以无阻塞 API 实现,便于异步任务模型下的使用。

在 ArceOS 的适配过程中,我们主要完成以下工作:
\begin{itemize}
    \item \textbf{设备抽象与适配}:通过 \texttt{DeviceWrapper} 实现对底层驱动的封装,并在 smoltcp 中实现相应的 \texttt{phy::Device} 接口;
    \item \textbf{协议裁剪机制}:通过 Feature 控制选择启用 TCP、UDP 等功能模块,按需裁剪协议栈体积;
    \item \textbf{内存集成管理}:采用静态缓冲池与 ArceOS 的 memory pool 深度集成,提高内存复用效率;
    \item \textbf{异步轮询接口}:提供 \texttt{poll()} 接口驱动数据包收发,适配 ArceOS 的异步调度机制。
\end{itemize}

在资源紧张或对延迟敏感的场景中,如 IoT 终端或边缘节点,smoltcp 的部署效果更加稳定高效。

\textbf{多协议栈并列设计优势分析} \par
将 lwIP 和 smoltcp 作为可选协议栈集成至 axnet 框架具有以下优势:
\begin{itemize}
    \item \textbf{运行时灵活切换}:系统在初始化阶段可根据用户配置或应用类型选择默认协议栈,灵活应对多样化部署场景;
    \item \textbf{面向场景适配}:高负载服务器推荐使用 lwIP 以获得高吞吐量;而 IoT 等轻量场景可启用 smoltcp 降低资源占用;
    \item \textbf{简化上层开发}:对上层统一暴露标准 API,协议栈变更对应用层透明,降低移植成本;
    \item \textbf{提升可测试性与健壮性}:协议栈间可对比测试验证行为一致性,提升系统整体稳定性。
\end{itemize}

\section{统一网络系统调用接口设计}
为实现协议栈无关的套接字操作,系统定义了统一的 syscall 接口,并通过 \texttt{SocketInner} 枚举持有协议栈具体实现。

\textbf{系统调用接口及功能描述} \par
系统通过 syscall ID 分发实现以下主要网络操作:
\begin{multicols}{2}
\begin{itemize}
    \item \texttt{SOCKET}
    \item \texttt{SOCKETPAIR}
    \item \texttt{BIND}
    \item \texttt{LISTEN}
    \item \texttt{ACCEPT}, \texttt{ACCEPT4}
    \item \texttt{CONNECT}
    \item \texttt{SENDTO}, \texttt{SENDMSG}, \texttt{SENDMMSG}
    \item \texttt{RECVFROM}
    \item \texttt{GETSOCKNAME}, \texttt{GETPEERNAME}
    \item \texttt{SETSOCKOPT}, \texttt{GETSOCKOPT}
    \item \texttt{SHUTDOWN}
\end{itemize}
\end{multicols}

\begin{listing}[!ht]
  \captionsetup{name=代码}
  \begin{minted}{rust}
  pub enum SocketInner {
      Lwip(LwipSocket),
      Smoltcp(SmoltcpSocket),
  }
  \end{minted}
  \caption{SocketInner 枚举封装协议栈实现}
  \label{lst:SocketInnerEnum}
\end{listing}

系统每次调用 \texttt{syscall()} 通过 match 分发到具体实现,隐藏协议栈底层细节。

\textbf{线程安全与条件编译机制} \par
为了支持 SMP 架构与多核并发访问,所有 socket 操作均封装在互斥锁 \texttt{Mutex} 内部,通过 Rust 安全抽象保证线程安全。协议栈选择通过 Cargo feature 标志实现裁剪与构建:

\begin{listing}[!ht]
  \captionsetup{name=代码}
  \begin{minted}{toml}
  [features]
  use-lwip = []
  use-smoltcp = []
  \end{minted}
  \caption{Cargo.toml 中的 feature 定义}
  \label{lst:CargoFeatures}
\end{listing}

\textbf{系统的可扩展性与未来支持} \par
该框架的设计支持未来拓展其他网络协议栈,如 nanonet、uIP 或 Rust-native 实现,支持多租户网络栈隔离、用户态网络协议实现、或高速 NIC 的零拷贝接入等。


