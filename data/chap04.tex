\chapter{网络管理组件设计与实现}

本章将围绕 ArceOS 系统下对网络管理组件的进一步设计与实现展开论述,
旨在增强其对 Linux 应用中广泛使用的 socket 网络编程模型的支持,
从而提升系统的网络兼容能力与实用性。

\section{开发环境与工具配置}
本章所描述的网络管理组件在 ArceOS 宏内核架构下完成,其开发环境与前述章节保持一致:以 Rust 语言为主,结合 Cargo 构建系统与模块依赖管理机制,
提供组件化的开发流程。

项目支持包括 RISC-V、x86-64、AArch64 和 LoongArch64 在内的多种硬件架构,
在 QEMU 模拟器平台上完成交叉编译与调试。
通过构建 Makefile 脚本,支持不同架构下对网络协议栈(如 lwIP、smoltcp)的动态选择与功能测试。

为辅助网络功能的开发与验证,使用 Wireshark 进行数据包捕获分析,结合 Netcat 与 iperf3 工具模拟常见的客户端/服务端通信场景,对 socket 接口的兼容性与性能进行实际验证。

\section{模块总体设计思路}

网络管理组件的设计目标是提供统一的、具备 Linux 语义兼容性的 socket 抽象,
支持 TCP 与 UDP 两种主要传输协议,同时保持 lwIP 与 smoltcp 协议栈的可插拔性。
其核心架构设计围绕如下几个要点:

\textbf{Socket 抽象与文件接口整合} \par
ArceOS 将 socket 抽象统一封装为 Socket 枚举类型,内部以 Mutex 保护底层 TCP/UDP socket,
确保在多线程环境下的并发访问安全。

该 Socket 类型同时实现了 FileLike 接口,与系统文件描述符机制对接,使得 socket 可被视为一种特殊的文件资源,从而统一了内核中不同 I/O 接口的调用方式。
这意味着标准的文件操作(如 read/write)可以直接用于 socket,系统调用层可以使用统一入口处理来自用户空间的文件与网络请求。

此外,该设计还为未来扩展如 unix domain socket、raw socket 等提供了良好基础,仅需在枚举中增加新变体并对接口进行实现即可完成协议扩展。

\textbf{系统调用支持与语义还原} \par
为满足 Linux 应用对 socket 的调用习惯,网络组件实现了如下关键系统调用接口:

\begin{itemize}
  \item \texttt{sys\_socket, sys\_bind, sys\_listen, sys\_accept, sys\_connect}
  \item \texttt{sys\_send, sys\_recv, sys\_sendto, sys\_recvfrom}
  \item \texttt{sys\_getsockname, sys\_getpeername, sys\_setsockopt, sys\_shutdown}
\end{itemize}

每个系统调用都遵循相似的处理流程:解析传参、校验合法性、调用 \texttt{Socket::from\_fd()} 映射底层资源,最后执行 socket 上对应方法。
部分调用(如 \texttt{sys\_sendto} / \texttt{sys\_recvfrom} )还需要处理地址结构与字节缓冲区的转换,并正确管理内存访问权限。

同时,通过复用 Rust 中 Result 类型与 axerrno 提供的错误码结构,网络模块能够优雅处理各类系统错误,保持与 Linux 错误语义对齐。

\textbf{协议栈解耦与多栈选择机制} \par
ArceOS 通过 feature 条件编译特性,实现对 lwIP 和 smoltcp 网络协议栈的动态替换。

在内核配置文件中用户可选择使用 lwIP 或 smoltcp,两者实现了统一的 trait 接口,使得在 socket 层操作中无需关注底层协议栈差异。

此外,在网络栈初始化过程中,系统会根据配置自动加载对应协议栈,初始化设备接口(如 TAP 网卡或 virtio 网卡),完成链路层与上层协议的绑定。

此机制的设计目标是为未来更多协议栈(如 embedded-nal 或 async-tcp)接入提供通用扩展点,实现灵活切换与无侵入式替换。

\section{lwIP 协议栈适配与封装实现}

\textbf{pbuf 管理机制解析} \par
lwIP 使用 pbuf 结构作为数据包缓冲区,其支持 $PBUF\_RAM$、$PBUF\_ROM$、$PBUF\_REF$ 等多种类型,分别对应内存分配方式与引用策略。

为提升性能,ArceOS 将 pbuf 的内存管理逻辑与内核中的 slab/heap 机制结合,统一使用内存池分配,以减少碎片并提升缓存命中率。

在数据收发过程中,通过 $pbuf\_chain$ 构建链式结构,实现分片组包能力;释放时则统一使用 $pbuf\_free$ 并增加引用计数机制确保内存安全。

同时,针对 Rust 所需的内存安全与生命周期管理,引入回调钩子机制,允许在 pbuf 释放时调用预设释放器,如关闭等待队列、释放关联任务上下文等。

\textbf{回调系统与事件驱动模型} \par
为了实现事件驱动模型,ArceOS 为 lwIP 的回调接口增加了适配器层:

\begin{itemize}
  \item TCP 接收事件通过 \texttt{tcp\_recv} 注册回调,在数据到达时将任务唤醒
  \item 连接建立通过 \texttt{tcp\_accept} 处理新连接,将新 \texttt{socket} 加入监听队列
  \item 发送完成事件通过 \texttt{tcp\_sent} 通知发送缓冲区空闲,可继续写入数据
  \item 错误事件通过 \texttt{tcp\_err} 回调报告异常,配合 shutdown 或自动回收资源
\end{itemize}

这些回调均结合 WaitQueue 与任务调度机制,在 socket 层实现阻塞与唤醒模型,兼顾 Linux 应用对同步 socket 行为的依赖。

通过事件注册与回调响应机制,lwIP 协议栈能够与 ArceOS 高效协作,完成连接建立、数据传输、异常关闭等完整连接生命周期管理。

\section{smoltcp 协议栈异步接口适配}

\textbf{poll 接口与非阻塞语义} \par
smoltcp 作为轻量化协议栈,其核心设计即为事件驱动,其接口暴露的 poll() 方法用于主动轮询网络接口状态。

ArceOS 在实现中为每个 TCP/UDP socket 包装状态机,每次调用 poll() 返回当前 socket 是否可读/可写,并与 poll/select/epoll 系统调用的语义一致对齐。

每个 Socket 对象实现 PollState 查询接口,并在网络事件发生后通过中断或轮询唤醒对应任务线程,实现非阻塞式的数据收发机制。

\textbf{结合任务调度的异步适配} \par
为适配 Rust 中广泛使用的 async/await 模型,ArceOS 在 socket 层引入 Waker 机制。

每当 socket 状态变更(如缓冲区变为可读、可写),协议栈会触发注册的 waker,唤醒阻塞的异步任务。

Waker 的注册通过 Future trait 实现中的 poll() 方法完成,结合 WakerQueue 实现多个任务对同一 socket 状态的订阅与唤醒。

该机制为 future-ready 的协议栈如 smoltcp 提供良好支持,为未来基于 async 编写的高性能网络服务打下基础。

\section{Socket 接口系统调用实现细节}
本节给出若干典型 socket 系统调用在 ArceOS 中的实现过程。

\subsection{sys\_socket}
用于创建新的套接字对象,根据传入的 domain、type、protocol 参数实例化对应类型的 socket,并将其加入文件描述符表。

\begin{figure}[htbp]
  \centering
  \includegraphics[width=0.5\textwidth]{figures/sys_socket.png}
  \caption{sys\_socket 系统调用的实现流程}
  \label{fig:sys_socket}
\end{figure}

\subsection{sys\_bind, sys\_connect, sys\_listen, sys\_accept}
这些系统调用封装了 socket 生命周期中的典型状态转换:
\begin{itemize}
  \item \texttt{bind} 将 socket 绑定至本地地址
  \item \texttt{connect} 主动连接远端主机
  \item \texttt{listen} 转为被动监听状态
  \item \texttt{accept} 从连接队列中提取 socket,返回新连接句柄
\end{itemize}

\begin{figure}[htbp]
  \centering
  % 第一张图片
  \begin{minipage}[t]{0.49\textwidth} % [t] 顶部对齐
    \centering
    \includegraphics[width=\linewidth, height=8cm]{figures/sys_bind.png} % 固定高度
    \caption{sys\_bind 系统调用的实现流程}
    \label{fig:sys_bind}
  \end{minipage}
  \hfill
  % 第二张图片
  \begin{minipage}[t]{0.49\textwidth} % [t] 顶部对齐
    \centering
    \includegraphics[width=\linewidth, height=10cm]{figures/sys_connect.png} % 相同高度
    \caption{sys\_connect 系统调用的实现流程}
    \label{fig:sys_connect}
  \end{minipage}
\end{figure}
  

\subsection{sys\_sendto / sys\_recvfrom}
这些调用支持 UDP 无连接数据报通信,允许在未连接状态下向任意地址发送数据。
调用时会尝试临时 bind,确保套接字处于就绪状态。

\begin{figure}[htbp]
  \centering
  % 第一张图片
  \begin{minipage}[t]{0.49\textwidth} % [t] 顶部对齐
    \centering
    \includegraphics[width=\linewidth, height=10cm]{figures/sys_sendto.png} % 固定高度
    \caption{sys\_sendto 系统调用的实现流程}
    \label{fig:sys_sendto}
  \end{minipage}
  \hfill
  % 第二张图片
  \begin{minipage}[t]{0.49\textwidth} % [t] 顶部对齐
    \centering
    \includegraphics[width=\linewidth, height=9cm]{figures/sys_recvfrom.png} % 相同高度
    \caption{sys\_recvfrom 系统调用的实现流程}
    \label{fig:sys_recvfrom}
  \end{minipage}
\end{figure}

\subsection{sys\_setsockopt 与选项控制}
用于设置 TCP/UDP 层面参数,如接收缓冲区大小、\texttt{TCP\_NODELAY} 等。
当前版本支持基本的结构识别与参数存取,未来可扩展为完整 Linux socket option 支持集。

\begin{figure}[htbp]
  \centering
  \includegraphics[width=0.7\textwidth]{figures/sys_setsockopt.png}
  \caption{sys\_setsockopt 系统调用的实现流程}
  \label{fig:sys_setsockopt}
\end{figure}

\section{组件之间的协作与系统集成}
网络组件在系统中与多个模块交互密切:

\begin{itemize}
  \item 与任务调度组件配合实现阻塞与唤醒逻辑
  \item 与内存管理模块共享缓存区与内存池资源
  \item 与文件系统统一接口实现 read/write/stat 多态行为
\end{itemize}

具体流程见图 \ref{fig:network_component_flow},该图展示了 socket 创建、数据发送接收、连接管理等操作的协同工作。
\begin{figure}[htbp]
    \centering
    \includegraphics[width=0.6\textwidth]{figures/net_flow.pdf}
    \caption{网络组件与系统其他模块的协作流程}
    \label{fig:network_component_flow}
\end{figure}

\section{小结}
本章对 ArceOS 网络管理组件的设计与实现进行了系统性阐述,从系统调用支持、socket 接口封装、协议栈适配到异步语义整合,全面覆盖了宏内核下对网络通信子系统的构建过程。未来可进一步支持 IPv6、raw socket、TLS 加密通道等高级网络特性,完善网络子系统的功能与性能。
